\documentclass[letterpaper,12pt,oneside,final]{book}
%%
%%  Template de mémoire de maîtrise ou thèse de doctorat.
%%  Normalement, il n'est pas nécessaire de modifier ce document
%%  sauf pour changer les noms des fichiers à inclure.
%%
%%  Version: 2014-10-28
%%
%%  Accepte les caractères accentués dans le document (UTF-8).
\usepackage[utf8]{inputenc}
%%
%% Support pour l'anglais et le français (français par défaut).
%\usepackage[cyr]{aeguill}
\usepackage{lmodern}      % Police de caractères plus complète et généralement indistinguable visuellement de la police standard de LaTeX (Computer Modern).
\usepackage[T1]{fontenc}  % Bon encodage des caractères pour qu'Acrobat Reader reconnaisse les accents et les ligatures telles que ffi.
\usepackage[english,frenchb]{babel} % le langage par défaut est le dernier de la liste, c'est-à-dire français
%%
%% Charge le module d'affichage graphique.
\usepackage{graphicx}
\usepackage{epstopdf}  % Permet d'utiliser des .eps avec pdfLaTeX.
%%
%% Recherche des images dans les répertoires.
\graphicspath{{./images/}{./dia/}{./gnuplot/}}
%%
%% Un float peut apparaître seulement après sa définition, jamais avant.
\usepackage{flafter,placeins}
%%
%% Utilisation de natbib pour les citations et la bibliographie.
\usepackage{natbib}
%%
%% Autres packages.
\usepackage{amsmath,color,soulutf8,longtable,colortbl,setspace,ifthen,xspace,url,pdflscape}
%% AJOUTÉ PAR LOIC -- PACKAGES
\usepackage{tikz}
%\usepackage{subcaption}
\usepackage[Algorithme]{algorithm}
\usepackage{algorithmic}
\usepackage{tabularx}
\usepackage{mathtools}
\usepackage{xspace}
\usepackage{amssymb}
\usepackage{subcaption}
\usepackage{soul}
\usepackage{amsthm}
\usepackage{tabularx}
\usepackage[frenchb,english]{babel}
\captionsetup{compatibility=false}
%%
%% Support des acronymes.
\usepackage[nolist]{acronym}
\onehalfspacing                % Interligne 1.5.
%%
%% Définition d'un style de page avec seulement le numéro de page à
%% droite. On s'assure aussi que le style de page par défaut soit
%% d'afficher le numéro de page en haut à droite.
\usepackage{fancyhdr}
\fancypagestyle{pagenumber}{\fancyhf{}\fancyhead[R]{\thepage}}
\renewcommand\headrulewidth{0pt}
\makeatletter
\let\ps@plain=\ps@pagenumber
\makeatother
%%
%% Module qui permet la création des bookmarks dans un fichier PDF.
%\usepackage[dvipdfm]{hyperref}
\usepackage{hyperref}
\usepackage{caption}  % Hyperlien vers la figure plutôt que son titre.
\makeatletter
\providecommand*{\toclevel@compteur}{0}
\makeatother
%%
%% Définitions spécifiques au format de rédaction de Poly.
\usepackage{MemoireThese}
%%
%% Définitions spécifiques à l'étudiant.
%% -----------------------------------
%% ---> A MODIFIER PAR L'ETUDIANT <---
%% -----------------------------------
%%
%% Commandes qui affichent le titre du document, le nom de l'auteur, etc.
\newcommand\monTitre{Opportunisme et ordonnancement en optimisation sans-dérivées}
\newcommand\monPrenom{Loïc Anthony}
\newcommand\monNom{Sarrazin-Mc Cann}
\newcommand\monDepartement{mathématiques et génie industriel}
\newcommand\maDiscipline{mathématiques appliquées}
\newcommand\monDiplome{M}        % (M)aîtrise ou (D)octorat
\newcommand\anneeDepot{2018}
\newcommand\moisDepot{avril}
\newcommand\monSexe{M}           % "M" ou "F"
\newcommand\PageGarde{O}         % "O" ou "N"
\newcommand\AnnexesPresentes{N}  % "O" ou "N". Indique si le document comprend des annexes.
\newcommand\mesMotsClef{Liste,de,mot-clés,séparés,par,des,virgules}
%%
%%  DEFINITION DU JURY
%%
%%  Pour la définition du jury, les macros suivantes sont definies:
%%  \PresidentJury, \DirecteurRecherche, \CoDirecteurRecherche, \MembreJury, \MembreExterneJury
%%
%%  Toutes les macros prennent 4 paramètres: Sexe (M/F), Prénom, Nom, Titres
\newcommand\monJury{\PresidentJury{M}{Prénom}{NOM}{Doct.}\\
\DirecteurRecherche{F}{Prénom}{Nom}{Ph.~D.}\\
\MembreJury{M}{Prénom}{Nom}{Ph.~D.}}

\ifthenelse{\equal{\monDiplome}{M}}{
\newcommand\monSujet{Mémoire de maîtrise}
\newcommand\monDipl{Maîtrise ès sciences appliquées}
}{
\newcommand\monSujet{Thèse de doctorat}
\newcommand\monDipl{Philosophi\ae{} Doctor}
}
%%
%% Informations qui sont stockées dans un fichier PDF.
\hypersetup{
  pdftitle={\monTitre},
  pdfsubject={\monSujet},
  pdfauthor={\monPrenom{} \monNom},
  pdfkeywords={\mesMotsClef},
  bookmarksnumbered,
  pdfstartview={FitV},
  hidelinks,
  linktoc=all
}
%%
%% Il y a un document par chapitre du mémoire.
%%
%% AJOUTÉ PAR LOIC
\usetikzlibrary{patterns}
\bibliographystyle{plain}
\newcommand{\D}{\mathbb{D}}
\newcommand{\N}{\mathbb{N}}
\newcommand{\Q}{\mathbb{Q}}
\newcommand{\R}{\mathbb{R}}
\newcommand{\norm}[1]{\left\lVert#1\right\rVert}
\newcommand{\Pset}{\mathcal{P}}
\newcommand{\CS}{\textsf{CS}\xspace}
\newcommand{\GPS}{\textsf{GPS}\xspace}
\newcommand{\MADS}{\textsf{MADS}\xspace}
\newcommand{\GSS}{\textsf{GSS}\xspace}
\newcommand{\imfil}{\textsf{IMFIL}\xspace}
\newcommand{\PS}{\textsf{PS}\xspace}
\newcommand{\POLL}{\textsf{POLL}\xspace}
\newcommand{\SEARCH}{\textsf{SEARCH}\xspace}
\makeatletter
\def\zz\ignorespaces{\@ifnextchar-{}{\phantom{-}}}
\newcolumntype{C}{>{\zz}{c}}
\makeatother
\newlength\myindent
\setlength\myindent{2em}
\newcommand\bindent{%
	\begingroup
	\setlength{\itemindent}{\myindent}
	\addtolength{\algorithmicindent}{\myindent}
}
\newcommand\eindent{\endgroup}
\DeclarePairedDelimiter{\normone}{||}{||_{1}}
\DeclarePairedDelimiter{\norminf}{||}{||_{\infty}}
\DeclarePairedDelimiter{\normemp}{||}{||}
\AtBeginDocument{\def\labelitemi{$\bullet$}}
\theoremstyle{theorem}
\newtheorem{definition}{Définition}[section]
\begin{document}
%%
%% Page de titre du mémoire.
\frontmatter
% Compte optionellement la page de garde dans la pagination.
\ifthenelse{\equal{\PageGarde}{O}}{\addtocounter{page}{1}}{}
\thispagestyle{empty}%
\begin{center}%
\vspace*{\stretch{1}}
UNIVERSITÉ DE MONTRÉAL\\
\vspace*{\stretch{1}}
\MakeUppercase{\monTitre}\\
\vspace*{\stretch{1}}
\MakeUppercase{\monPrenom~\monNom}\\
DÉPARTEMENT DE \MakeUppercase{\monDepartement}\\
ÉCOLE POLYTECHNIQUE DE MONTRÉAL\\
\vspace*{\stretch{1}}
\ifthenelse{\equal{\monDiplome}{M}}{MÉMOIRE PRÉSENTÉ}{THÈSE PRÉSENTÉE} EN VUE DE L'OBTENTION\\
DU DIPLÔME DE \MakeUppercase{\monDipl}\\
(\MakeUppercase{\maDiscipline})\\
\MakeUppercase{\moisDepot} \anneeDepot
\end{center}%
\vspace*{\stretch{1}}
\copyright~\monPrenom~\monNom, \anneeDepot.
%%
%% Identification des membres du jury.
%%
\newpage\thispagestyle{empty}%
\begin{center}%
\vspace*{\stretch{2}}
\ul{UNIVERSITÉ DE MONTRÉAL}\\
\vspace*{\stretch{1}}
\ul{ÉCOLE POLYTECHNIQUE DE MONTRÉAL}\\
\vspace*{\stretch{2}}
Ce\ifthenelse{\equal{\monDiplome}{M}}{~mémoire intitulé}{tte thèse intitulée}:\\
\vspace*{\stretch{1}}
\MakeUppercase{\monTitre}\\
\vspace*{\stretch{2}}
\end{center}%
\begin{flushleft}
présenté\ifthenelse{\equal{\monDiplome}{M}}{}{e}
par:~\ul{\mbox{\MakeUppercase{\monNom} \monPrenom}}\\
en vue de l'obtention du diplôme de:~\ul{\mbox{\monDipl}}\\
a été dûment accepté\ifthenelse{\equal{\monDiplome}{M}}{}{e} par le jury d'examen constitué de:\end{flushleft}
\vspace*{\stretch{2}}
\monJury
%%
\pagestyle{pagenumber}%
%% Dédicace
%%
%% La dédicace est un hommage que l'auteur souhaite
%% rendre à une ou plusieurs personnes de son choix.
%%
\chapter*{DÉDICACE}\thispagestyle{headings}
\addcontentsline{toc}{compteur}{DÉDICACE}
\begin{flushright}
  \itshape
 	\ldots
\end{flushright}
          % Dédicace du document.
% Remerciements
%
%   Grâce aux remerciements, l'auteur attire l'attention du lecteur
% sur l'aide que certaines personnes lui ont apportée, sur leurs
% conseils ou sur toute autre forme de contribution lors de la
% réalisation de son mémoire. Le cas échéant, c'est dans cette section
% que le candidat doit témoigner sa reconnaissance à son directeur de
% recherche, aux organismes dispensateurs de subventions ou aux
% entreprises qui lui ont accordé des bourses ou des fonds de
% recherche.
\chapter*{REMERCIEMENTS}\thispagestyle{headings}
\addcontentsline{toc}{compteur}{REMERCIEMENTS}
%
Texte.
     % Remerciements.
% Résumé du mémoire.
%
%   Le résumé est un bref exposé du sujet traité, des objectifs visés,
% des hypothèses émises, des méthodes expérimentales utilisées et de
% l'analyse des résultats obtenus. On y présente également les
% principales conclusions de la recherche ainsi que ses applications
% éventuelles. En général, un résumé ne dépasse pas quatre pages.
%
%   Le résumé doit donner une idée exacte du contenu du mémoire ou de la thèse. Ce ne
% peut pas être une simple énumération des parties du document, car il
% doit faire ressortir l'originalité de la recherche, son aspect
% créatif et sa contribution au développement de la technologie ou à
% l'avancement des connaissances en génie et en sciences appliquées.
% Un résumé ne doit jamais comporter de références ou de figures.
\chapter*{RÉSUMÉ}\thispagestyle{headings}
\addcontentsline{toc}{compteur}{RÉSUMÉ}

Le résumé est un bref exposé du sujet traité, des objectifs visés,
des hypothèses émises, des méthodes expérimentales utilisées et de
l'analyse des résultats obtenus. On y présente également les
principales conclusions de la recherche ainsi que ses applications
éventuelles. En général, un résumé ne dépasse pas quatre pages.

Le résumé doit donner une idée exacte du contenu du mémoire ou de la thèse. Ce ne
peut pas être une simple énumération des parties du document, car il
doit faire ressortir l'originalité de la recherche, son aspect
créatif et sa contribution au développement de la technologie ou à
l'avancement des connaissances en génie et en sciences appliquées.
Un résumé ne doit jamais comporter de références ou de figures.
      % Résumé du sujet en français.
% Abstract
%
%   Résumé de la recherche écrit en anglais sans être
% une traduction mot à mot du résumé écrit en français.

\chapter*{ABSTRACT}\thispagestyle{headings}
\addcontentsline{toc}{compteur}{ABSTRACT}
%
\begin{otherlanguage}{english}

Written in English, the abstract is a brief summary similar to the previous
section {\selectlanguage{frenchb}(Résumé)}.  However, this section is not a
word for word translation of the French.

\end{otherlanguage}
          % Résumé du sujet en anglais.

{\setlength{\parskip}{0pt}
%%
%% Table des matières.
\renewcommand\contentsname{TABLE DES MATIÈRES}
\tableofcontents
%%
%% Liste des tableaux.
\renewcommand\listtablename{LISTE DES TABLEAUX}
\listoftables
%%
%% Table des figures.
\renewcommand\listfigurename{LISTE DES FIGURES}
\listoffigures
%%
%% Liste des annexes au besoin.
}

% Liste des sigles et abbréviations
\newcommand\abbrevname{LISTE DES SIGLES ET ABRÉVIATIONS}
\chapter*{\abbrevname}
\addcontentsline{toc}{compteur}{\abbrevname}
\pagestyle{pagenumber}
%
\begin{acronym}
  \acro{IETF}{Internet Engineering Task Force}
  \acro{OSI}{Open Systems Interconnection}
\end{acronym}
%
\begin{longtable}{lp{5in}}
\textsf{CS}       & Coordinate Search\\
\textsf{GPS}       & Generalized Pattern Search\\
\textsf{MADS}       & Mesh Adaptive Direct Search\\
\textsf{GSS}       & Generating Set Search\\
\textsf{imfil}       & Implicit Filtering\\
\textit{BBO}       & Blackbox optimisation\\
\textit{DFO}       & Derivative-Free Optimization\\
\end{longtable}

       % Liste des sigles et abréviations.
% Liste des sigles et abbréviations
\newcommand\symbname{LISTE DES SYMBOLES}
\chapter*{\symbname}
\addcontentsline{toc}{compteur}{\symbname}
\pagestyle{pagenumber}
%
\begin{acronym}
  \acro{IETF}{Internet Engineering Task Force}
  \acro{OSI}{Open Systems Interconnection}
\end{acronym}
%
\begin{longtable}{lp{5in}}
$B$ & Matrice du modèle quadratique\\
$C^k$ & Matrice génératrice.\\
$c$ & Contrainte quantifiable\\
$D$ & Un ensemble générateur positif.\\
$d$ & Direction de recherche\\
$d_s$ & Profil de donnée du solveur $s$.\\
$e_i$ & Direction coordonnée.\\
$F^k$ & Cadre à l'itération $k$.\\
$f$ & Fonction objectif.\\
$f_{\Omega}$ & Fonction objectif sous barrière extrême.\\
$G$ & Matrice de base.\\
$H^k$ & Ensemble des directions supplémentaires pour la \textsf{SEARCH}.\\
$h$ & Quantification de la violation de contraintes\\
$k$ & Compteur d'itérations.\\
$L^k$ & Matrice de directions complémentaires à l'itération $k$.\\
$l$ & Borne inférieur\\
$M^k$ & Maillage à l'itération $k$.\\
$n$ & Dimension du problème.\\
$P^k$ & Ensemble des points de sonde à l'itération $k$. \\
$P$ & Ensemble de problèmes.\\
$p$ & Problème\\
$S^k$ & Ensemble des points de sonde à l'itération $k$. \\
$S$ & Ensemble de solveurs\\
$s$ & Solveur\\
$u$ & Borne supérieure\\
$x^k$ & Centre de sonde à l'itération $k$.\\

$\beta$ & Borne sur la longueur d'une direction.\\
$\Gamma^k$ & Matrice génératrice de l'espace.\\
$\Delta^k$ & Longueur du cadre à l'itération $k$\\
$\delta^k$ & Longueur du pas à l'itération $k$\\
$\epsilon_{\text{stop}}$ & Critère d'arrêt de l'algorithme\\
$\kappa(D^k)$ & Mesure cosinus de l'ensemble $D^k$.\\
$\lambda^k$ & Paramètre de contraction.\\
$\xi$ & Élément de la base naturelle.\\
$\rho(\delta^k)$ & Fonction de force.\\
$\rho_s$ & Profil de performance du solveur $s$.\\ 
$\tau$ & Paramètre d'ajustement du maillage.\\
$\phi^k$ & Paramètre d'expansion \\
$\Omega$ & Ensemble réalisable de solution\\
\end{longtable}

       % Liste des sigles et abréviations.
\ifthenelse{\equal{\AnnexesPresentes}{O}}{\listofappendices}{}
\mainmatter
%   Dans l'introduction, on présente le problème étudié et les buts
% poursuivis. L'introduction permet de faire connaître le cadre de la
% recherche et d'en préciser le domaine d'application. Elle fournit
% les précisions nécessaires en ce qui concerne le contexte de
% réalisation de la recherche, l'approche envisagée, l'évolution de
% la réalisation. En fait, l'introduction présente au lecteur ce
% qu'il doit savoir pour comprendre la recherche et en connaître la
% portée.
\Chapter{INTRODUCTION}\label{sec:Introduction}  % 10-12 lignes pour introduire le sujet.
\section{Optimisation de boîtes noires}\label{sec:bbo}
L'optimisation est l'étude de l'obtention d'un minimum ou d'un maximum pour un problème donné. Les problèmes sont définis par leur vecteur de variables $x$, la fonction objectif sous-jacente $f(x)$ ainsi que leur ensemble réalisable $\Omega$. On cherche alors à résoudre :
\begin{gather}
\underset{x}{\min}\{f(x)\ :\ x\ \in\ \Omega\}.
\end{gather}
avec $f : \R ^n \rightarrow \R$ et $\Omega \in R^n$. Ce problème peut prendre plusieurs formes, et plusieurs disciplines de l'optimisation existent en conséquent. Les problèmes peuvent être tels que $x$ est composé de variable continues, entières ou une combinaison des deux, pour lesquels existent l'optimisation continue, en nombres entiers et mixte. Une grande quantité de spécificités sur les variables, la fonction objectif est les contraintes mènent à une panoplie de sous-disciplines en optimisation.\\
Les principaux algorithmes d'optimisation exploitent la structure même du problème, telle la connaissance des dérivées du premier et du second degré pour la résolution. On pense ici à la méthode du simplexe pour l'optimisation linéaire~\cite{NaSo96a}, ou encore de les méthodes de Newton ou de Quasi-Newton pour l'optimisation non-linéaire~\cite{NoWr99a}.\\
Dans le cas traité dans cet ouvrage, on étudie un certain type de problème dit \textbf{boîte noire}. Les boîtes noires sont caractérisées par leur fonctionnement; elles prennent en entrée un vecteur de variables, et en ressort un résultat. Les fonctions sous-jacentes sont inconnues, trop complexes ou trop instables pour en retirer une forme analytique. On peut résumer une boîte noire à un problème complexe instable, demandant beaucoup de ressources à résoudre, dont les optima locaux sont fréquents et dont les dérivées sont inconnues, difficilement obtenables ou même inexistante. Un problème de boîte noire est en réalité l'optimisation d'une fonction qui est un boîte noire, et dont les contraintes sont aussi évaluées par des boîtes noires.\\
La discipline de l'optimisation qui se concentre sur la résolution de problèmes dont la structure de la fonction objectif et des fonctions contraintes ne peux être exploitée se nomme l'optimisation de boîte noire (\textit{BBO - Blackbox Optimization}). On distingue de cette discipline l'optimisation sans dérivées (\textit{DFO -  Derivative-free optimization}), dans laquelle on étudie la résolution de problème d'optimisation en n'utilisant aucune valeurs de dérivées de la fonction~\cite{AuHa2018,AuKok2016} Ainsi, les outils de \textit{DFO} utilisent seulement les valeurs de la fonction objectif calculées afin de bâtir des modèles qui seront eux-mêmes utilisés dans le processus d'optimisation. Contrairement à la \textit{BBO}, la \textit{DFO}  ne sous-entends pas l'inexistence des dérivées de la fonction à optimiser, seulement qu'elles ne sont peut-être pas appropriées comme outil pour la résolution du problème.\\
Conn, Scheinberg et Vincente~\cite{CoScVibook} regroupent les méthodes de \textit{DFO}  en deux familles :  les méthodes de recherche directe et les méthodes de régions de confiance. Les méthodes de recherche directe se résument à ce que d'autres nomment les méthodes d'échantillonnage (sampling methods)~\cite{Kelley2011} : l'optimisation est contrôlée par l'évaluation de $f(x)$ dans un ensemble de point, et c'est le résultat de ces évaluations qui détermine le prochain ensemble de point, jusqu'à la convergence. Les méthodes de régions de confiance, quant à elles, utilisent des modèles pour déterminer des points candidats, évaluent ces points et mettent à jour les modèles avec les résultats de ces points pour ensuite trouver un autre point candidat et répéter ce processus jusqu'à la convergence.
\section{Définition de la problématique}\label{sec:dpr}
Les méthodes de recherche directe diffèrent des méthodes de régions de confiance de par leur structure. Plusieurs définitions existent pour les méthodes de recherche directe. Hooke et Jeeves~\cite{HoJe61a} proposent en 1961 une définition qui éclaircie un peu l'idée derrière les méthodes.
\begin{quote}
	On utilise l'expression «recherche directe» pour décrire l'examination séquentielle de solutions candidates impliquant la comparaison de chaque candidat à la meilleure obtenue précédemment, avec une stratégie pour déterminer les prochains candidats [...].
\end{quote}
Depuis l'avènement de problèmes dont les dérivées sont pratiquement inexistantes ou incalculables, les méthodes de recherche directe sont devenues au cœur des intérêts de plusieurs chercheurs. Cependant, puisque les évaluations sont couteuses, il y a la naissance d'un besoin de développer ces méthodes de façon à conserver leur fondements théoriques de convergence tout en ayant recours à un minimum d'évaluations de la fonction objectif. Par exemple, Audet et al.~\cite{AuIaLeDTr2014} ont travaillé à la génération d'un espace générateur avec le moins de directions possible pour ainsi réduire le nombre de directions nécessaires à chaque itération.
À chaque itération, un algorithme de recherche directe possède une liste d'au moins un point candidat, tandis qu'un algorithme de régions de confiance n'en a assurément qu'un seul. Certaines méthode de recherche directe sont cependant similaires aux méthodes de régions de confiance en ce sens, tel que l'algorithme de Nelder Mead~\cite{NeMe65a}, qui est classée comme une méthode de recherche directe simpliciale par Conn, Scheinberg et Vincente~\cite{CoScVibook}.\\
Puisque chaque évaluation de la fonction objectif est très couteuse en ressources, on en vient à se demander si l'évaluation de chaque point dans la liste générée à chaque itération est nécessaire au bon déroulement de l'algorithme. Torczon~\cite{Torc97a} démontre qu'une famille d'algorithme de recherche directe converge vers un optimum sans avoir à évaluer tous les points dans une liste générée à une itération donnée. Ainsi, la convergence n'est pas assurée par la qualité de la direction de descente utilisée à chaque itération $f(x^k+\delta^k d) = \min(f(x^k+\delta^k d:d\in D)) $ , mais par sa propriété de répondre à la définition de décroissance simple $f(x^k+\delta^k d)~-~f(x^k)~<~0$, avec $\delta^k$ la longueur du pas et $d \in D^k $ la direction de descente choisie dans l'ensemble des directions $D$, pour l'itération $k$. On en vient à l'élaboration d'une problématique qui justifie le projet. Pour un algorithme possédant à l'itération $k$ une solution courante $x_k$ et un liste de candidat $P^k$, on cherche à déterminer si la découverte d'un nouveau meilleur point dans cette liste devrait entrainer un passage prématuré à l'itération suivante de l'algorithme.\\
\section{Objectif de la recherche} \label{sec:obj}
Le long de ce mémoire, l'interruption prématurée d'une étape nécessaire à la convergence d'un algorithme sera nommée \textit{stratégie opportuniste} ou \textit{opportunisme}, de l'expression pour référer à cette nomenclature introduite par Coope et Price~\cite{CoPr01a} et reprise par Audet et Dennis~\cite{AuDe04a}. La question restant très vague, on devra définir plusieurs aspects afin d'en venir à une quantification de l'impact de l'opportunisme sur le déroulement d'algorithmes de \textit{DFO}.\\
En premier lieu, la structure de l'algorithme doit admettre qu'au cours d'une itération il existe une liste de point à évaluer séquentiellement avant la détermination d'une autre liste de points. Ainsi, appartenir à la famille des méthodes de recherche directe n'est pas un critère suffisant compte tenu que l'algorithme de Nelder Mead y figure, malgré qu'il évalue un point à la fois. Cependant, d'autres famille d'algorithmes ouvrent la porte à l'opportunisme par leur fabrication. Notons ici les méthodes de recherche linéaire basée sur les dérivées simplexe, telles que baptisées par Conn, Scheinberg and Vincente~\cite{CoScVibook}, qui ne concordent pas exactement à la définition de méthodes de recherche directe. Un objectif premier de cette recherche est d'identifier les algorithmes laissant place à l'implémentation de l'opportunisme en eux.\\
De l'utilisation de l'opportunisme apparait une autre problématique au coeur de cette recherche, c'est à dire la stratégie utilisée pour ordonner les points candidats dans la liste. Si on permet une interruption de la séquence d'évaluation des points sur la liste lors de l'obtention d'un meilleur point, alors l'ordre dans laquelle apparaît ces points sera au cœur du comportement de l'algorithme. On nommera la règle définissant l'ordre des points dans la liste la \textit{stratégie d'ordonnancement}. Le second objectif de cette recherche est d'identifier les stratégies d'ordonnancement qui pourraient avoir un impact sur la performance de la stratégie opportuniste.\\
Ayant en main un éventail d'algorithmes et de stratégies d'ordonnancement, on pourra procéder à l'objectif principal de la recherche, soit la quantification de la performance de l'utilisation de la stratégie opportuniste dans les algorithmes, ainsi que la comparaison des différentes stratégies d'ordonnancement identifiées. On cherche ainsi à déterminer les cas dans lesquels l'opportunisme est une stratégie envisageable, nécessaire, ou contraignante. On cherche aussi à caractériser les différentes stratégies d'ordonnancement en fonction du type de problème à résoudre, des spécificités algorithmiques ou en fonction de la précision demandée à l'algorithme. 
\section{Plan du mémoire}\label{sec:pla}
       % Introduction au sujet de recherche.
\Chapter{OUTILS D'OPTIMISATION SANS-DÉRIVÉES}\label{sec:Theme1}
Cette section du travaille consiste en un état de l'art sur les outils utilisés en optimisation sans-dérivées par recherche directe. On se concentrera sur la description des outils qui sont nécessaires à la définition et à l'application de la stratégie algorithmique étudiée, soit l'opportunisme, décrite à la section \ref{sec:Theme2}. En premier lieu, une revue des méthodes de recherche directe ciblées pour l'étude est effectuée. Cette étape nécessite une grande attention à l'identification des caractéristiques des méthodes qui garantissent la convergence théorique de la méthode. De cette façon, les modifications apportées aux algorithmes pourront être effectuées sans interférer avec les hypothèses obligatoires pour l'analyse de convergence. Ensuite, on traitera de la gestion des contraintes en fonction de leurs types, de façon à observer subséquemment les interactions entre l'opportunisme et les techniques de gestion de contraintes à notre disposition. Enfin, on définira les modèles quadratiques utilisés par quelques unes des implémentations utilisées lors des essais numériques de la section \ref{sec:Theme3}.
\section{Coordinate Search}\label{sec:cs}
Le premier algorithme à être abordé est aussi le plus intuitif pour l'optimisation sans contraintes, qu'on appelera \emph{Coordinate Search} (\CS{}), ou Recherche par coordonnée, parfois appelée \emph{Compass Search}~\cite{KoLeTo03a}. On attribue à Fermi et Metropolis~\cite{FeMe1952} cette première méthode de recherche directe. Pour que leur modèle suive bien leur ensemble de données expérimentales sur la diffusion nucléaire, Fermi et Metropolis ont fait varier les paramètre théoriques de déphasage de leur fonction un à la fois avec un pas constant. Lorsque ni l'augmentation ni la diminution de l'un des paramètres améliorait la concordance avec les données expérimentales, la longueur du pas était diminuée de moitiée et le processus était recommencé. On continuait ainsi jusqu'à ce que le pas soit considéré suffisement petit. C'est ainsi qu'est née la méthode de la recherche par coordonnée. 
Booker et al.~\cite{BoDeFrSeToTr99a} proposent un cadre rigoureux aux algorithmes de recherche de motifs. On y stipule que les algorithmes doivent être divisés en deux étapes, soient la recherche qu'on notera \SEARCH{} et la sonde qu'on notera \POLL{} afin d'être conforme avec la littérature. La \SEARCH{} consiste à l'implentation d'une méthode visant à explorer le domaine de la fonction sans égard à la détermination d'un optimum. Le \POLL{} cherche à minimiser la fonction pour déterminer un optimum. Dans le cadre qui nous intéresse, le \POLL{} serait l'entièreté du processus, tandis que la \SEARCH{} n'y trouverait pas son correspondant. La description de l'algorithme \CS{} décrite en vertu du cadre proposé par Booker et al. est fortement inspirée de celle de Audet et Hare, issue de~\cite{AuHa2018}.
\begin{algorithm}[H]
	\caption{\textsf{Recherche par coordonnée} (\CS)}
	\label{alg:cs}
	\begin{algorithmic}
		\STATE Avec $f:\R^n \rightarrowtail \R$ la fonction objectif et $x^0$ le point de départ
		\STATE 0. \textsf{Initialisation des paramètres} : 
		\bindent
		\STATE\begin{flushleft}
			\begin{tabular}{l l}
				$\delta^0 \in (0,\infty)$ & la longueur du pas initial\\
				$\epsilon_{\text{stop}} \in \left[ 0,\infty \right) $ & le critère d'arrêt\\
				$k \leftarrow 0$ & le compteur d'itérations\\
			\end{tabular}
		\end{flushleft}
		\eindent
		\STATE 1. \POLL
		\bindent
		\IF {$f(t) < f(x^k) $ pour un $t \in P^k := \{x^k \pm \delta^k e_i : i = 1,2,\dots,n \}$}
		\STATE $x^{k+1} \leftarrow t$ et $\delta^{k+1} \leftarrow \delta^k$
		\ELSE
		\STATE $x^{k+1} \leftarrow t$ et $\delta^{k} \leftarrow \frac{1}{2}\delta^k$
		\ENDIF
		\eindent
		\STATE 2. \textsf{Terminaison}
		\bindent
		\IF {$\delta^k \geq \epsilon_{\text{stop}}$}
		\STATE $k\leftarrow k+1$
		\STATE go to 1.
		\ELSE
		\STATE stop
		\ENDIF
		\eindent
	\end{algorithmic}
\end{algorithm}
À l'étape \POLL de l'algorithme, on entend que la fonction $f(x)$ est évaluée à chaque élément de $t \in P^k$ avant de déterminer quel $t$ deviendra le nouveau centre de sonde $x^{k+1}$. Cependant, si les évaluations sont effectuées en série, on aura une séquence de points à évaluer. La séquence proposée est celle de $2n$ mouvements dans les directions elementaires suivie de la détermination d'un nouveau centre de sonde si au moins une évaluation est un succès. C'est dans cette séquence que l'opportunisme pourra être introduit, afin de ne pas évaluer le reste de la liste si le test de décroissance simple est positif.
\section{Generalized Pattern Search}\label{sec:gps}
Le deuxième algorithme présenté est la recherche par motifs généralisée. La premiere recherche par motifs en soit est élaboré par Hooke et Jeeves~\cite{HoJe61a}. Ils nomment la recherche par motifs (\emph{Pattern Search} ou \PS) la routine de recherche directe visant à minimiser une fonction $f(x)$ avec leur algorithme. Cette routine est composée d'une série de mouvements autour d'un point qui peuvent être divisés en deux types, soit les mouvements exploratoires (\emph{Exploratory Moves}) ou les mouvements destinés à la détermination d'un minimum, soit les mouvements de motifs (\emph{Pattern Moves}). Les mouvements exploratoires  servent à la détermination du motifs, c'est à dire au comportement de la fonction $f(x^k)$ aux alentours d'un point $x^k$. Les mouvements de motifs sont ensuite effectués dans la direction que les mouvements exploratoires ont déterminé comme étant celle qui se dirige vers le minimum de la fonction. Hooke et Jeeves introduisent aussi la définition de succès et d'échec. Un succès est mouvement tel que la valeur de $f(x)$ au point est inférieure à la meilleure connue auparavant. Dans le cas contraire, on dira que c'est un échec.\\
Lors de l'élaboration de cette technique ils mentionnent que :
\begin{quote}
	Par soucis de simplicité, les mouvements exploiratoires sont choisis de façon simple, c'est à dire, à chaque mouvement seulement la valeur d'une unique coordonée est changée.
\end{quote}
Pour ensuite affirmer que : 
\begin{quote}
	Suivant un mouvement de motif fructueux, il est raisonnable de conduire une série de mouvements exploratoires et de tenter d'améliorer d'avantage les résultats.
\end{quote}
De ces deux affirmations découlent les principes de bases de \PS{}, c'est à dire une sucscession de mouvements exploratoires dans les direction coordonnées $e_i : i =\pm \{1,2\dots n\}$ et d'un mouvement de motifs dans la meilleure direction. Chaque succès de la recherche de motifs entraine que la prochaine serie de mouvements exploratoires sera effectuée autour de ce nouveau meilleur point. Lorsque la succession échoue, la longueur du pas est réduite afin de déterminer un nouveau motif. Pour préciser la méthode, un point initial $x^0$ doit être déterminé ainsi qu'une longueur de pas initiale $\delta^0$ et un critère $\epsilon_{\textsf{stop}}$ pour lequel on jugera que le pas est devenu suffisement petit. De plus, on doit déterminer la méthode utilisée pour procéder à la réduction de la longueur du pas. Dans les mots de Hooke et Jeeves, le \POLL{} serait l'ensemble de mouvements exploratoires accompagné de la mise à jour du centre de sonde, tandis que la \SEARCH{} serait le mouvement de motif restant.\\
Torczon~\cite{Torc97a} amène une généralistation du \PS{} de Hooke et Jeeves. Dans cet article, l'auteure approche la méthode d'un autre oeil. Le motif propre à l'itération $P^k$ est issu de la multiplication de deux matrices $GC^k$, soient $G \in \R^{n\times n}$ la matrice de base et $C^k \in \R^{n\times p}, p > 2n$, la matrice generatrice. La matrice de base se doit d'être non-singulière et la matrice génératrice se doit d'être composée telle que $C^k = [\Gamma^k~ L^k]$, ou $\Gamma^k$ est la concaténation d'une matrice de plein rang $M$ et de son opposée. Le rôle de $\Gamma^k$ est de générer l'espace $\R ^n$, tandis que le rôle de la $L^k$ est de complémenter cette derniere avec des directions supplémentaires ne servant pas à strictement à générer l'espace, rappelant la \SEARCH{}. Ainsi, $GC^k$ est une matrice qui génére l'espace $\R^n$ et qui se libère du cadre restreignant des directions unitaires proposé par Fermi et Metropolis et repris par Hooke et Jeeves. Armés de $GC^k$ et d'une longueur de pas spécifique à une itération $\delta^k$, on retrouve $\delta^k Gc^k_i$, soit une généralisation de $\delta^k e_i$ présent dans \CS{}, mais pour lequel on a une colonne $c^k_i\in C^k$ perturbée par $G$ remplaçant la direction elementaire.\\
Toujours selon Torczon, la définition de mouvements exploratoires est reprises pour être plus générale. Un mouvement exploratoire est en fait un vecteur issu du motif $s^k \in P^k$. L'auteure demande aussi que, si il existe un vecteur colonne $c^k_i \in P^k$ tel que $f(x^k + \delta^k c_i^k) < f(x^k)$, alors les mouvements exploratoires doivent produire un pas $s^k$ tel que $f(x^k+s^k) < f(x^k)$.\\
L'algorithme s'inscrit alors en cinq étapes. Premièrement, à l'initialisation de l'algorithme, vient le calcul de la fonction $f(x)$ à l'itéré initial $x^0$ à l'itération $0$. Deuxièmement, le calcul d'une direction $s_k$ issue de la série de mouvements exploratoires. Troisièment, le calcul de $f(x^k + s^k)$. Ensuite vient la mise à jour $x^{k+1} = x^k +s^k$ si l'étape précédente est un succès, sinon $x^{k+1} = x^k$. Enfin, au besoin, on  met à jour l'ensemble générateur $C^k$ et la longueur du pas $\delta^k$ et on recommence le processus à partir de a deuxième étape jusqu'à ce que $\delta^k$ soit jugé suffisamment petit.\\
Afin de rester dans le cadre de Booker et al.~\cite{BoDeFrSeToTr99a}, on dépeint l'algoritme en se rattachant aux concepts de \POLL{} et de \SEARCH{}. Les directions issues de la matrice supplémentaire $f(x^k + s^k), s^k \in GL^k$ introduite par Torczon~\cite{Torc97a} correspondent à une étape de \SEARCH{} en vertu de sa fonction d'incorporer des directions supplémetaires à la recherche, tandis que les mouvements exploratoires visant à trouver un minimum autour du point de sonde ($f(x^k + s^k), s^k \in D\Gamma^k$) correspondent à la section \POLL{}. Cependant, on généralisera d'avantage $D\Gamma^k$ afin de le remplacer par un ensemble générateur positif~\cite{Davi54b}, noté $D=GZ, Z\in \R ^{n\times p}$, qui réponds aux exigence exactes énoncées dans~\cite{Torc97a} sans imposer que la taille de la matrice bornée supérieurement à $p = 2n$, mais bien tel que $n+1 \leq p \leq 2n$, en concordance avec les propriétés d'une base positive. Cette représentation est fortement inspirée de celle fournie par Audet et Hare dans~\cite{AuHa2018}. On dira de $S^k =GL^k$ qu'il est l'ensemble des directions supplémentaires $L^k$ fournit à l'itération $k$ soumit à une transformation par la matrice inversible  $G \in \R ^{n \times n}$.\\
Par ailleurs, il est à noter que les algorithmes de recherche directe énoncés dans~\cite{LeTo96b,Torc97a,LeTo99a,LeTo00a}, qui sont des algorithmes déclinant de la famille d'algorithmes \GPS, diffèrent à la formulation donnée dans ce document inspirée de Audet et Hare~\cite{AuHa2018}, notamment sur un aspect principal. La mise à jour de la longueur du pas se fait par deux paramètre différents, soient $\delta^{k+1} =\phi^k \delta^k, \phi^k \geq 1$ pour l'expansion du maillage dans le cas ou l'itération $k$ est un succès et $\delta^{k+1} =\lambda^k \delta^k, \lambda^k \in (0,1)$ dans le cas d'un échec. Dans la formulation présente on utilise un seul paramètre $\tau \in (0,1)$ et son inverse $\tau^{-1}$ pour cette mise à jour. Cependant, cette altération ne contrevient pas aux requis de la démonstration de la convergence de~\cite{Torc97a} qui assurent que la formulation présente possède les même propriétés. La formulation présente aussi une différence notable avec celle fait par Booker et al. dans~\cite{BoDeFrSeToTr99a}, c'est à dire l'admission de la mise à jour de la longueur du pas pour une \SEARCH fructueuse, alors que Booker et al. incorporent la mise à jour seulement dans un échec de \POLL, à l'image de l'utilisation d'un facteur $\phi^k=1$.
\begin{algorithm}[H]
	\caption{\textsf{Recherche par motifs généralisée} (\GPS)}
	\label{alg:gps}
	\begin{algorithmic}
		\STATE Avec $f:\R^n \rightarrowtail \R$ la fonction objectif et $x^0$ le point de départ
		\STATE 0. \textsf{Initialisation des paramètres} : 
		\bindent
		\STATE\begin{flushleft}
			\begin{tabular}{l l}
				$\delta^0 \in (0,\infty)$ & la longueur du pas initial\\
				$D=GZ$ & un ensemble générateur positif\\
				$\tau \in (0,1) \cup \Q$ & le paramètre d'ajustement du maillage\\
				$\epsilon_{\text{stop}} \in \left[ 0,\infty \right) $ & le critère d'arrêt\\
				$k \leftarrow 0$ & le compteur d'itérations\\
			\end{tabular}
		\end{flushleft}
		\eindent
		\STATE 1. \SEARCH
		\bindent
		\IF {$f(t) < f(x^k) $ pour un $t \in S^k$ } %\subseteq M^k$}
		\STATE $x^{k+1} \leftarrow t$ et $\delta^{k+1} \leftarrow \tau ^{-1}\delta^k$
		\STATE go to 3
		\ELSE
		\STATE go to 2.
		\ENDIF
		\eindent
		\STATE 2. \POLL
		\bindent
		\STATE détermination d'un ensemble générateur positif $\D^k \subseteq \D$
		\IF {$f(t) < f(x^k) $ pour un $t \in P^k := \{x^k + \delta^k d : d \in D^k\}$}
		\STATE $x^{k+1} \leftarrow t$ et $\delta^{k+1} \leftarrow \tau ^{-1}\delta^k$
		\ELSE
		\STATE $x^{k+1} \leftarrow t$ et $\delta^{k} \leftarrow \tau\delta^k$
		\ENDIF
		\eindent
		\STATE 3. \textsf{Terminaison}
		\bindent
		\IF {$\delta^k \geq \epsilon_{\text{stop}}$}
		\STATE $k\leftarrow k+1$
		\STATE go to 1.
		\ELSE
		\STATE stop
		\ENDIF
		\eindent
	\end{algorithmic}
\end{algorithm}
L'opportunisme pourra être implémenté dans l'étape de \POLL{} selon la même logique que dans \CS, c'est à dire à même la liste des points $x^k + \delta^k d : d \in D^k$. Pour ce qui est de la \SEARCH, l'implémentation de la stratégie est faisable mais sa performance n'en vient qu'à dépendre de la pertinence de la méthode pour générer $S^k$.
\section{Mesh Adaptive Direct Search}\label{sec:mad}
Le troisième algorithme présenté est dans les même lignée que \CS et \GPS. Il s'agit de la recherche direct sur maillage adaptif (\emph{Mesh Adaptive Direct Search} ou \MADS), formulé par Audet et Dennis~\cite{AuDe2006}. Contrairement aux algorithmes précédents, \MADS est un algorithme basé sur un maillage, c'est à dire que chaque évaluation de la fonction objectif se fera à un point sur une discrétisation de l'espace des variables. On dénotera $M^k:=\{x^k + \delta^kDy : y \in \N^p\} \subset \R^n$ le maillage à l'itération $k$ sur lequel les itérés seront définis, généré par l'ensemble générateur positif $D = GZ$ et $y$ les entiers naturels. Il s'agit ici d'un maillage de cardinalité infini. Dans leur formulation de Audet et Hate reprise ici, les algorithmes \CS et \GPS sont aussi considéré comme étant basé sur un maillage, quoique leurs premières références~\cite{HoJe61a,Torc97a} n'en indique pas autant.\\
La grande distinction de \MADS ne réside alors pas dans son utilisation d'un maillage comme structure, mais bien l'incorporation d'un cadre à son étape de sonde. On définit le cadre tel que $F^k:=\{x\in M^k : \norminf{x-x^k} \leq \Delta^k b \}$ à l'itération $k$, soit l'ensemble des points sur le maillage $M^k$ pour lesquels la distance de Chebyshev à l'itéré courant est inférieur à une valeur $\Delta ^k b$. Cette valeur est déterminée par $b = \max\{\norminf{d'}:d' \in \D\}$, soit la plus grande composante vectorielle présente dans tous les colonnes de la base positive $\D$ issue de l'ensemble générateur $D$, multipliée par un paramètre introduit dans \MADS, soit le paramètre de longueur du cadre $\Delta^k$.\\
La figure \ref{fig:MADS} montre une itération hypothétique de \MADS pour laquelle $\Delta^k = \frac{1}{2}, \delta^k = \frac{1}{8}$ avec l'ensemble générateur étant
$D = \begin{bmatrix*}[C]
1 & 0 &-1\\
0 & 1 & -1
\end{bmatrix*}$. \\
\begin{figure}[H] %FIGURE  : MESH
	\begin{center}
		\begin{tikzpicture}
		% Flèches
		\draw [very thin,gray!50] (0,0) grid[step=0.5] (5,5);
		\draw [very thick] (0.5,0.5) rectangle (4.5,4.5);
		\draw [->,thick] (2.5,2.5)  -- (4,1) node [below left,scale=0.65]{$x^k+\delta^k d_1 $}; 
		\draw [->,thick] (2.5,2.5) -- (1,2.5) node [below,scale=0.65]{$x^k+\delta^k d_2 $};
		\draw [->,thick] (2.5,2.5) -- (3,4.5) node [below right,scale=0.65]{$x^k+\delta^k d_3 $};
		\draw (2.5,2.5) node [above right,scale=0.65]{$x^k$};
		\end{tikzpicture}
	\end{center}
	\caption{Ensemble générateur, maillage et cadre}
	\label{fig:MADS}
\end{figure}
Dans \GPS, quoique l'algorithme décrit n'inclue pas de maillage formellement, les évaluations sont limitées aux directions de $D$ mise à l'échelle avec la longueur du pas. Dans \MADS, on pourra choisir n'importe quelle direction telle que $x^k + \delta^k d$,tant que l'ensemble des directions $d$ forment un ensemble générateur positif $D_\Delta^k$ qui soit compris dans le cadre $F^k$. La mise à jour de la longueur du cadre se fait avec le paramètre d'ajustement du maillage qui, dans \CS et \GPS, servait  pour la mise à jour de $\delta^k$. Dans \MADS, $\delta^k$ est mis à jour par défaut est d'utiliser $\delta^k = \min(\Delta^k, (\Delta^k)^2)$. Ainsi, on s'assure de respecter $\delta^k \leq \Delta^k$ et on augmente exponentiellement le nombre de directions possibles si $\Delta^k \leq 0$. Afin de rester dans le cadre de Booker et al.~\cite{BoDeFrSeToTr99a}, on décrit l'algorithme en se rattachant aux concepts de \POLL et \SEARCH avec un étape \SEARCH limitée à un ensemble $S^k$ de points.  La description de l'algorithme est fortement inspirée de celle de Audet et Hare, issue de~\cite{AuHa2018}. 
\begin{algorithm}[H]
	\caption{\textsf{Recherche par treillis adaptifs} (\MADS)}
	\label{alg:mad}
	\begin{algorithmic}
		\STATE Avec $f:\R^n \rightarrowtail \R$ la fonction objectif et $x_0$ le point de départ
		\STATE 0. \textsf{Initialisation des paramètres} : 
		\bindent
		\STATE\begin{flushleft}
			\begin{tabular}{l l}
				$\Delta^0 \in (0,\infty)$ & la longueur du cadre initial\\
				$D=GZ$ & un matrice génératrice positive\\
				$\tau \in (0,1) \cup \Q$ & le paramètre d'ajustement du maillage\\
				$\epsilon_{\text{stop}} \in \left[ 0,\infty \right) $ & le critère d'arrêt\\
				$k \leftarrow 0$ & le compteur d'itérations\\
			\end{tabular}
		\end{flushleft}
		\eindent
		\STATE 1. \textsf{Mise à jour des paramètres}
		\bindent
		\STATE $\delta^k \leftarrow \min(\Delta^k,(\Delta^k)^2)$
		\eindent
		\STATE 2. \SEARCH
		\bindent
		\IF {$f(t) < f(x^k) $ pour un $t \in S^k$ } %\subseteq M^k$}
		\STATE $x^{k+1} \leftarrow t$ et $\Delta^{k+1} \leftarrow \tau ^{-1}\Delta^k$
		\STATE go to 4
		\ELSE
		\STATE go to 3
		\ENDIF
		\eindent
		\STATE 3. \POLL
		\bindent
		\STATE avec le cadre $F^k$ de demi-coté $\Delta^k$
		\STATE détermination d'un ensemble générateur positif $\D^k_\Delta \subset F^k$
		%\STATE tel qu'il est un sous-ensemble du cadre $F^k$ de demi-coté $\Delta^k$.
		\IF {$f(t) < f(x^k) $ pour un $t \in P^k := \{ x^k + \delta^k d : d \in \D ^k_\Delta\}$}
		\STATE $x^{k+1} \leftarrow t$ et $\delta^{k+1} \leftarrow \tau ^{-1}\Delta^k$
		\ELSE
		\STATE $x^{k+1} \leftarrow t$ et $\delta^{k} \leftarrow \tau\Delta^k$
		\ENDIF
		\eindent
		\STATE 4. \textsf{Terminaison}
		\bindent
		\IF {$\delta^k \geq \epsilon_{\text{stop}}$}
		\STATE $k\leftarrow k+1$
		\STATE go to 1.
		\ELSE
		\STATE stop
		\ENDIF
		\eindent
	\end{algorithmic}
\end{algorithm}
L'implémentation de la stratégie opportuniste sera faite à l'étape de \POLL selon la même logique que pour \CS et \GPS. 
\section{Generating Set Search}\label{sec:gss}
Le quatrième algorithme est la recherche par ensemble générateur. Il est introduit par Kolda, Lewis et Torczon dans~\cite{KoLeTo03a} sous le nom de \textit{Generating Set Search} qu'on abrègera \GSS. Il s'agit d'un algorithme très similaire à \GPS mais qui incorpore certains aspects laissés de côté lors de la formulation de \GPS et certains aspects nouveaux propre à \GSS. À l'instar de \GSS, un ensemble de direction à chaque itération est dédié à la \SEARCH, soit $H^k$ et un ensemble dédié à la \POLL, soit $D^k$.

La différence principale entre \GSS et \GPS est le critère d'acceptation d'un itéré comme étant un succès. Pour les algorithmes précédents, chaque itération entrainant une simple diminution $f(x^k+\delta^kd)<f(x^k)$ était considérée comme un succès. Dans \GSS, on permet que le critère d'acceptation soit resserré avec l'addition du terme $\rho(\delta^k)$ au test de diminution, soit que $f(x^k+\delta^kd)<f(x^k)+\rho(\delta^k)$. $\rho(\delta)$ sera appelée la fonction de force et doit satisfaire à une des deux définitions. Soit $\rho(\delta)$ est continue, $\rho(\delta) = o(\delta)$ lorsque $\delta \downarrow 0$ et que $\rho(\delta_1) < \rho(\delta_2)$ si $\delta_1 < \delta_2$, ou soit $\rho(\delta)\equiv0$. Le premier cas impose une diminution suffisante de la fonction, rappelant le critère d'Armijo~\cite{Armi66a,griva2009linear}. Cet outil permet d'assurer la convergence globale théorique de l'algorithme en utilisant les résultats de convergence de Lucidi et Sciandrone~\cite{SLucidi_MSciandrone_2002} avec seulement une fonction objectif $f(x)$ bornée inférieurement. Le deuxième cas corresponds à une condition de diminution simple. Sous cette condition, les résultats de convergence globale sont assurés par les démonstrations faites sur les algorithmes basés sur treillis. Les conditions de convergences sont resserrés, de façon à ce que \GSS convergera dans le cas où les paramètres fournies restreignent les points candidats à un maillage, et qu'aucune expansion de ce maillage n'est admise.~\cite{KoLeTo03a,CoPr01a}. Il est à noter que \CS et \GPS sont des algorithmes de la famille de \GSS, c'est à dire qu'une paramétrisation spécifique de \GSS permets d'obtenir \GPS ou \CS.  
  
Les auteurs spécifient deux autres grande différences en comparaison avec \CS. Premièrement, on y fait valoir l'utilisation d'un ensemble générateur positif comme ensemble de directions de sonde, un aspect déjà incorporé dans notre définition de \GPS. L'autre différence se situe dans la souplesse des paramètres de contraction $\tau^k$ et des paramètres d'expansion $\phi^k$.\\
Plusieurs paramètres supplémentaires sont nécessaires tels que la mesure du cosinus, le plus petit angle entre deux directions de l'ensemble générateur $\kappa(G^k)$ qui empêche le choix de mauvaises directions, à l'instar de la mesure d'angle~\cite{griva2009linear,OrRh70a} en recherche linéaire. On y fait mention aussi de la longueur des vecteurs de l'ensemble générateur étant bornées tel que $\beta_{\min}<\norm{d}<\beta_{\max}, \beta_{\max} \geq \beta_{\min} >0$.
\begin{algorithm}[H]
	\caption{\textsf{Recherche par ensemble générateur} (\GSS)}
	\label{alg:gss}
	\begin{algorithmic}
		\STATE Avec $f:\R^n \rightarrowtail \R$ la fonction objectif et $x^0$ le point de départ
		\STATE 0. \textsf{Initialisation des paramètres} : 
		\bindent
		\STATE\begin{flushleft}
			\begin{tabular}{l l}
				$\delta^0 \in (0,\infty)$ & la longueur du pas initial\\
				$\lambda_{\max} \in (0,1) \cup \R$ & le paramètre de contraction maximale de la longueur du pas\\
				$\lambda^0 \in (0,1) \cup \R, \phi^0$ & les paramètre de contraction et d'expansion du pas\\
				\begin{tabular}{@{}l@{}}$\rho:[0,+\infty]\rightarrow\R$\\~
				\end{tabular} 
				&\begin{tabular}{@{}l@{}}une fonction continue tel que $\nabla(\rho(\delta)) < 0$ lorsque $\delta\rightarrow 0$\\
					et $\frac{\rho(\delta)}{\delta}\rightarrow 0$ quand $\delta \downarrow 0$\end{tabular} \\
				$\beta_{\max}\geq\beta_{\min}>0$ & les bornes sur la longueur des vecteurs de $G^k$\\
				$\epsilon_{\text{stop}} \in \left[ 0,\infty \right) $ & le critère d'arrêt\\
				$\kappa_{\min} \geq 0$ & La mesure cosinus minimale d'un ensemble\\
				$k \leftarrow 0$ & le compteur d'itérations\\
			\end{tabular}
		\end{flushleft}
		\eindent
		\STATE 1. \SEARCH
		\bindent
		\STATE détermination de $H^k = \{s \in \R^n,~\beta_{\min}\leq \norm{s}\}$
		\IF {$f(t) < f(x^k) - \rho(\delta^k) $ pour un $t \in S^k=\{x^k+\delta^ks, s \in H^k\}$ } %\subseteq M^k$}
		\STATE $x^{k+1} \leftarrow t$ et $\delta^{k+1} \leftarrow \phi^k \delta^k$
		\STATE go to 3
		\ELSE
		\STATE go to 2.
		\ENDIF
		\eindent
		\STATE 2. \POLL
		\bindent
		\STATE détermination de $D^k = \{s\in\R^n,~\beta_{\min}\leq\norm{s}\leq\beta_{\max},~\kappa(H^k\cup D^k)\geq \kappa_{\min}\} $
		\STATE un ensemble générateur
		\IF {$f(t) < f(x^k) $ pour un $t \in P^k := \{x^k + \delta^k d : d \in D^k\}$}
		\STATE $x^{k+1} \leftarrow t$ et $\delta^{k+1} \leftarrow \phi^k\delta^k$
		\ELSE
		\STATE $x^{k+1} \leftarrow t$ et $\delta^{k} \leftarrow \tau^k\delta^k$
		\ENDIF
		\eindent
		\STATE 3. \textsf{Terminaison}
		\bindent
		\IF {$\delta^k \geq \epsilon_{\text{stop}}$}
		\STATE $k\leftarrow k+1$
		\STATE go to 1.
		\ELSE
		\STATE stop
		\ENDIF
		\eindent
	\end{algorithmic}
\end{algorithm}
\section{Implicit Filtering}\label{sec:imf}
Le dernier algorithme est celui du filtrage implicite (\textit{Implicit Filtering} ou \imfil). Cette méthode, développée par Kelley~\cite{Kell99b,Kelley2011}, se veut un algorithme d'optimisation hybride, sans-contraintes explicites, de recherche linéaire se basant sur le gradient du stencil qui incorpore des particularités de \CS. Un stencil $V$ consiste en un ensemble de points répartis selon un motif autour d'un point central, par exemple $x^k$. Si le motif consiste en l'ensemble des directions coordonnées, le stencil prendra la forme suivante : 
 \begin{figure}[H] %FIGURE  : STENCIL
	\begin{center}
		\begin{tikzpicture}[
		scale=1,
		mydot/.style={
			circle,
			fill=white,
			draw,
			outer sep=0pt,
			inner sep=1.5pt
		}
		]
		% Flèches
		\draw [very thin,gray!50] (0,0) grid[step=0.5] (4,4);
		\draw [-,thick] (2,2) -- (3.5,2) node [mydot, scale=1.5]{};
		\draw [-,thick] (2,2) -- (0.5,2) node [mydot, scale=1.5]{};
		\draw [-,thick] (2,2) -- (2,3.5) node [mydot, scale=1.5]{};
		\draw [-,thick] (2,2) -- (2,0.5) node [mydot, scale=1.5]{};
		\draw (2,2) node []{$\bullet$};
		\draw (2,2) node [above right,scale=0.75]{$x^k$};
		\end{tikzpicture}
	\end{center}
	\caption{Stencil $V$ autour du point $x^k$ avec $n=2$}
	\label{fig:STENCIL}
\end{figure} On définira le gradient simplex de \imfil comme l'approximation du gradient de la fonction objectif $f$ en calculant la valeur de la fonction à chaque point d'un stencil et en posant :
\begin{gather*}
\nabla_{s}f(x^k) = \frac{1}{h^k}\delta(f,x^k,V,h)V^{\dagger}
\end{gather*}
avec $V\in\R^{n \times 2n}$ une matrice représentant les directions du stencil, $\delta(F,x,V,h^k)$ le vecteur colonne ayant $f(x+hv_j) - f(x)$ à son $j^e$ élément, $\{v_j\} \in V$, $V^{\dagger}$ son inverse de Moore-Penrose~\cite{GoVL1996} et $h^k$ le pas de l'itération. L'utilisation de la pseudo-inverse est justifiée par la nécessité de calculer des différences finies d'un seul côté si un point du stencil n'est pas réalisable. L'idée fondamentale de \imfil est de calculer le gradient stencil en un itéré courant $x^k$, avec des différences simples ou centrées, et d'utiliser le gradient simplex $\nabla_{s}f(x^k) \in \R^n$ comme direction pour effectuer une recherche linéaire à l'image d'une descente du gradient~\cite{NoWr2006}.  
  
L'algorithme appartient à la famille de méthodes sans-dérivées puisqu'il n'utilise pas le calcul de dérivées analytiques de la fonction. Cependant, contrairement aux étapes de \POLL des algorithmes présentés précédemment, l'algorithme approxime la dérivée à l'aide de différences finies. Pour déterminer les valeurs de $f(x+hv_i^1)$ servant au calcul du gradient simplex, l'auteur introduit une nomenclature proche de la notre, soit le \textsf{stencil poll}, qu'on généralisera sonde du stencil, qui est constituée de $n$ à $2n$ évaluations de la boîte noire, dans les cas opposés où deux contraintes de bornes sont actives et aucune contrainte de borne n'est active, si le motif utilisé est une combinaison des vecteurs unitaires de $\R^n$. C'est cette sonde du stencil qui sera analogue aux sections \POLL des algorithmes présentés antérieurement.

Soit $\tau > 0$ un paramètre de tolérance pour la terminaison sur la norme du gradient fournie par l'utilisateur. Suivant une sonde du stencil fructueuse, si la norme de l'approximation du gradient du stencil n'est pas plus petite que $\tau h^k$, l'algorithme effectue une recherche linéaire dans la direction opposée au gradient du stencil $ d = -\nabla_{s}f(x^k)$ avec un nombre de pas maximal spécifié par l'utilisateur. Alternativement, la direction peut-être calculée avec la résolution dy système d'équations linéaires $Hd = \nabla_{s}f(x^k)f(x^k)$, où $H \in \R^{n\times n}$ est une matrice Hessienne issue d'un modèle mis à jour selon une méthode de Quasi-Newton (par exemple Broyden-Fletcher-Goldfarb-Shanno (BFGS)~\cite{Broy65a,Flet65a}),$\nabla_{s}f(x^k) \in \R^n$ le gradient du stencil et $f(x^k)$ la fonction objectif évaluée à l'itéré courant. La recherche linéaire qui suit peut s'écrire de la façon suivante qu'on abrègera \textsf{BLS} pour \textit{Backtracking Line Search}. 
\begin{gather}
\underset{m}{\min}\{m~:~f(x^k+\beta^m d) < f(x^k),m \in \{0,\text{maxitarm}\}\cup\N\}.
\end{gather}
où $\beta< 1$ agit comme un facteur diminuant pour déterminer la longueur du pas nécessaire, $d \in \R^n$ la direction de descente, $m$ un entier servant de puissance à $\beta$ et  $\text{maxitarm}$ est le nombre maximal de pas de recherche linéaire imposé par l'utilisateur. L'idée ici est de trouver un $m$ minimal pour lequel la fonction évaluée en $f(x^k+\beta^m)<f(x^k)$.\\
Dans le cas ou la sonde échoue, la longueur du pas est diminuée de moitié et aucune autre recherche n'a lieu. L'algorithme est ainsi recommencé jusqu'à ce que la longueur du pas soit diminuée en deçà d'un seuil prescrit. La description de l'algorithme simplifié sort du cadre de travail précédemment imposé~\cite{BoDeFrSeToTr99a}. Par contre, il est possible de faire un parallèle entre celui-ci et \imfil. La sonde du stencil peut être vue comme l'étape de \POLL des méthodes de recherche directe, sur laquelle repose l'analyse de convergence. La sonde du stencil est suivie d'une recherche linéaire, une heuristique qui a pour but d'accélérer la convergence de l'algorithme. Un parallèle peut être fait entre la recherche linéaire de \imfil et une étape de \SEARCH des méthodes de recherche directe. Cependant, la recherche linéaire est détaillée et fixée, contrairement aux étapes de \SEARCH possibles pour les autres méthodes. Pour conclure l'analogie, on évoque qu'une itération de \imfil peut être interprétée comme une étape de \POLL suivie d'une étape de \SEARCH, qui celle-ci est conditionnelle au succès de la \POLL, à savoir la mise en marche d'une recherche linéaire seulement lorsque la sonde du stencil est fructueuse.
\begin{algorithm}[H]
	\caption{\textsf{Filtrage implicite} (\imfil)}
	\label{alg:imf}
	\begin{algorithmic}
		\STATE Avec $f:\R^n \mapsto \R$ la fonction objectif et $x^0$ le point de départ
		\STATE 0. \textsf{Initialisation des paramètres} :
		\bindent 
		\STATE\begin{flushleft}
			\begin{tabular}{l l}
				$h^0 \in (0,\infty)$ & la longueur du pas initial\\
				$V = \{\pm e_1,\pm e_2,\dots,\pm e_k\}$ & un stencil\\
				$\text{budget}$ & le nombre d'évaluations maximal\\
				$\text{maxitarm}$ & nombre maximal d'itérations de recherche linéaire\\
				$\epsilon_{\text{stop}} $ & le critère d'arrêt sur le pas\\
				$k \leftarrow 0$ & le compteur d'itérations de l'algorithme\\
				$\tau > 0$ & la tolérance admise sur le gradient\\
			\end{tabular}
		\end{flushleft}
		\eindent
		\STATE 1. \textsf{Boucle principale} :
		\bindent 
		\WHILE {$f_{\text{count}} < \textsf{budget} ~\&~ h^k > \epsilon_{\text{stop}}$}
			\STATE Effectuer la sonde du gradient
			\STATE Poser $P^k := \{x^k+h^kv_i: v_i \in V\}$ et choisir $x^{n+1}\in \arg\min\{f(x):x\in P^k \cup \{x^k\}\}$
			\STATE \textsf{Descente linéaire}
			\STATE Mise à jour de $H$ si nécessaire
			\STATE évaluer $-\nabla_{s}f(x^k)$
			\IF {$\norm{\nabla_{s}f(x^k)}\geq \tau h^k$ \& $x^{k+1}\neq x^k$}
				\STATE $d \leftarrow -\nabla_{s}f(x^k)$ ou $d \leftarrow H^{-1}\nabla_{s}f(x^k)f(x^k)$
				\IF {\textsf{BLS} est réalisable} 
					\STATE $x^{k+1} \leftarrow x^k + \beta^m d$
				\ENDIF
			\ELSE
			\STATE $h^{k+1} \leftarrow {h^k}/{2}$
			\ENDIF
		\ENDWHILE
		\eindent
	\end{algorithmic}
\end{algorithm}
\section{Gestion des contraintes en DFO}\label{sec:gcd}
Le problème (1) demande que $x$ soit compris dans l'ensemble $\Omega$. On peut représenter $\Omega$ de la façon suivante afin de représenter le cas ou les seules contraintes existantes sont des bornes : 
\begin{gather*}
\Omega = \{x\ \in\ \R^n\ :\ l_i \leq (x)_i \leq u_i\}\\
\end{gather*}
Où $l \in \R^*\cup\{-\infty\}$ représente le vecteur des bornes inférieures et $u \in \R^*\cup\{+\infty\}$ le vecteur des bornes supérieures. Sous cette même forme, on peut définir un problème comme étant non borné, si toutes les valeurs de $l$ et de $u$ prennent la valeur infinie, ce qui équivaut à $ X = \R ^n$ \\
Si la variable $x$, en plus d'être limitée par des bornes, est contrainte par un ensemble d'inégalité, tel que : 
\begin{equation*}
\begin{aligned}
\underset{x\in X \subset \R ^n}{\min}& & & f(x)\\
\text{s.t.}& & & c(x) \leq 0\\
\end{aligned}
\end{equation*}
On peut représenter son ensemble $\Omega$ réalisable tel que
\begin{gather*}
\Omega = \{x~\in~X~\subset \R ^n~:~ c(x) \leq 0\}
\end{gather*}
où $c:x\rightarrow \R ^n$. Une approche pour traiter ce problème se nomme \textbf{la barrière extrême}~\cite{AuDe2006}. On définit une fonction de barrière suivante
\begin{align*}
f_\Omega = \begin{cases}
f(x)~ &\text{si $x \in \Omega$}\\
\infty~ &\text{sinon}
\end{cases}
\end{align*}
On peut alors redéfinir la fonction à optimiser comme étant $f_\Omega$ en ne se souciant plus des contraintes. Cette approche ne prends cependant pas en compte la nature des contraintes. Dans cet ordre d'esprit, Le Digabel et Wild~\cite{LedWild2015} proposent une taxonomie des contraintes rencontrées en optimisation de boîtes noires. Dépendemment de la forme de la boîte noire, il est possible qu'une solution $f(x)$ existe, et ce même si l'évaluation de la contrainte est violée $c(x)<0$. Dans la terminologie de Le Digabel et Wild, on parles de contraintes de type \textsf{QR*K}, soit l'ensemble des contraintes connues, quantifiables et relaxables. On peut alors quantifier la violation de ces contraintes ainsi~\cite{AuHa2018} : 
\begin{align*}
h(x) = \begin{cases}
\sum_{j\in J}^{}(\max(c_j(x),0))^2~ &\text{si}~x\in X\\
\infty~ &\text{sinon}
\end{cases}
\end{align*}
Avec $J$ l'ensemble des indices des contraintes et $X$ l'ensemble des points. Ces contraintes seraient alors relaxables. Audet et Dennis~\cite{AuDe09a} proposent une façon de gérer la relaxation avec la \textbf{barrière progressive}. Afin de déterminer une hiérarchie des points selon leur valeur de $h(x)$ et $f(x)$, on définie des notions de dominance entre deux points. On dit que le point $x$ réalisable domine le point $y$ réalisable si l'expression suivante est respectée 
\begin{align*}
f(x) < f(y) \implies x\prec_{f} y ~ &\text{si}~x,y\in\Omega.\\
\end{align*}
Cependant, si $x$ est non-réalisable, il domine le point $y$ non-réalisable si
\begin{align*}
f(x) \leq f(y), h(x)\leq h(y) \implies x\prec_{h} y ~ &\text{si}~x,y\in\Omega/X
\end{align*}
avec au moins une inégalité stricte.

L'algorithme de la barrière progressive admet deux solutions courantes, soient $x^{feas}$ la solution réalisable dont la valeur de la fonction objectif est minimale, et $x^{inf}$, la solution irréalisable non-dominée dont la valeur de la fonction objectif est minimale et où $h(x^\text{feas})$ est inférieur à $h^k_{\max}$, un paramètre limite pour $h$ spécifique à l'itération $k$.

Dans la Figure ~\ref{fig:barrier}, on peut observer quelques points candidats. Les points pleins sont les points non-dominés et les points vides sont dominés ou exclus par la barrière $h_{\max}$. Ceux présent dans la zone grise sont dominés par des points irréalisables pour lesquels $f(x)$ ou $h(x)$ sont inférieurs,  ou encore leur valeur de leur fonction de violation $h(x)$ est supérieur à $h_{\max}$. L'interaction entre l'utilisation des concepts de barrière et les différents algorithmes d'optimisation sans dérivées sera vue lors de la description des algorithmes.
\begin{figure}[h]
	\begin{center}
		\begin{tikzpicture}
		% Flèches
		\draw[->] (0,0) -- (8,0);
		\draw [->] (0,0) -- (0,5);
		% Axes 
		\draw (8,0) node[right] {$h$};
		\draw (0,5) node[above] {$f$};
		% Point faisable non domin
		\draw (0,3.5) node[]{$\bullet$};
		\draw (0,3.5) node[below left]{$x^{feas}$};
		%Point faisable dominé
		\draw (0,4.5) node[]{$\circ$};
		% Points non-dominés mais pas inf
		\draw (2,3.9) node[]{$\bullet$};
		\draw (4,1.7) node[]{$\bullet$};
		% Point non-dominé inf
		\draw (6,0.7) node[]{$\bullet$};
		\draw (6,0.7) node[below left]{$x^{inf}$};
		% Points dominés
		\draw (2.9,4.2) node[]{$\circ$};
		\draw (5,1.7) node[]{$\circ$};
		\draw (6.6,1) node[]{$\circ$};
		\draw (7.5,0.5) node[]{$\circ$};
		% Droites delimitant zone realisable
		\draw [-] (2,5) -- (2,3.9);
		\draw [-] (2,3.9) -- (4,3.9);
		\draw [-] (4,3.9) -- (4,1.7);
		\draw [-] (4,1.7) -- (6,1.7);
		\draw [-] (6,1.7) -- (6,0.7);
		\draw [-] (6,0.7) -- (7,0.7);
		% Droite pour h max
		\draw [-] (7,0) -- (7,5);
		\draw (7,0) node[below]{$h_{max}$};
		% Hachure de la région dominée
		\fill[pattern = north east lines,opacity = 0.5] (2,5) -- (2,3.9) -- (4,3.9) -- (4,1.7) -- (6,1.7) -- (6,0.7) -- (7,0.7) -- (7,0) -- (8,0) --(8,5) -- cycle;
		\end{tikzpicture}
	\end{center}
\caption{Barrière progressive, dominance et solutions courantes multiples} \label{fig:barrier}
	\end{figure}
La portion de l'algorithme de la barrière progressive analogue à la sonde se déroule de la façon suivante. On utilise la nomenclature propre à \CS à des fins de simplification  :
\begin{itemize}
	\item Si il y a un point $t \in P^k$ qui domine la solution réalisable ou la solution irréalisable, alors on met à jour la solution nouvellement dominée. On mets à jour $\delta^k$ selon un cas de succès et on met à jour la barrière $h^{k+1}_{\max} \leftarrow h(t)$ si $t$ domine $x^{inf}$. Il s'agit d'une itération \emph{dominante}.
	\item Si aucun nouveau point dominant est trouvé, mais qu'on est en présence d'un point $t \in P^k$ pour lequel $0 < h(t) < h(x^{inf})$, alors on met à jour la barrière $h^{k+1}_{\max} \leftarrow h(t)$. Il s'agit d'une itération \emph{améliorante}.
	\item Sinon, on mets à jour $\delta^k$ selon un cas d'échec. Il s'agit d'un échec.
\end{itemize}
\section{Modèles quadratriques}\label{sec:mod}
	Conn et Le Digabel~\cite{CoLed2011} font mention de l'utilisation de modèles quadratiques en optimisation sans dérivées. Ces modèles peuvent servir de fonction substitut qui donne une approximation de la fonction et pour laquelle le coût de computation est réduit. Pour obtenir un modèle quadratique, on considère la base naturelle de l'espace des polynômes de degré deux et moins. 
	\begin{equation*}
	\xi (x)=(\xi_0(x),\xi_1(x),...,\xi_q(x))^T = \left(1,x_1,x_2,...,x_n,\frac{x_1^2}{2},\frac{x_2^2}{2},...,\frac{x_n^2}{2},x_1 x_2, x_1 x_3,...,x_{n-1},x_{n}\right)^T
	\end{equation*}
	Cette base possède $q+1 = (n+1)(n+2)/2$ éléments. Le modèle $\tilde{f}$ de la fonction $f$ est tel que $\tilde{f}(x)=\alpha^T\xi(x)$, $\alpha \in \R^{q+1}$. Pour obtenir ce modèle, un ensemble de point $Y=\{y^0,y^1,...,y^p\}$ ayant $p+1$ éléments est nécessaire. On cherche alors a minimiser la différence entre les valeurs de la boîte noire évaluées aux points de $Y$ et celles du modèle. Au long de l'exécution présente les algorithmes procéderont à l'évaluation de la boîte noire à différents points. Ces point la valeur de la fonction objectif évaluée à ceux-ci sont enregistré dans une cache, parmi laquelle pourront être choisis les $p = q$ points nécessaires à l'élaboration du modèle. Les points devront satisfaire la propriété qui valide le modèle : 
	\begin{gather*}
	B(\xi,Y)\alpha = f(Y)\\
	f(y)=(f(y^0,f(y^1),...,f(y^p))^T\\
	B(\xi,Y) = 
	\begin{bmatrix}
	\xi_0(y^0) & \xi_1(y^0) & \dots & \xi_q(y^0)\\
	\xi_0(y^1) & \xi_1(y^1) & \dots & \xi_q(y^1)\\
	\vdots & \vdots & \vdots & \vdots\\
	\xi_0(y^p) & \xi_1(y^p) & \dots & \xi_q(y^p)\\
	\end{bmatrix}_.
	\end{gather*}
	Ce système peut être résolu seulement si $p=q$ et si la matrice est de rang pleine. Dans le cas où $p\geq q$, on tentera de résoudre le problème de minimisation suivant : 
	\begin{equation*}
	\begin{aligned}
	& \underset{\alpha \in \R}{\text{min}}
	& & \norm{B(\xi,Y)\alpha -f(Y)}^2
	\end{aligned}.
	\end{equation*}
	Dans le cas où $p<q$, on minimisera le même problême mais en régularisant le problème à l'aide d'une interpolation dans le sense de la norme de Frobenius minimale~\cite{MoWi2009,CuRoVi10}. La norme de Frobenius pour une matrice $A$ est définie par : 
	\begin{equation*}
	\norm{A}_F = \sqrt{\overset{m}{\underset{i=1}{\sum}} \overset{n}{\underset{j=1}{\sum}}|a_{i,j}|}.
	\end{equation*}
	Il est possible de réarranger $\tilde{f}(x)$ de façon à l'illustrer comme une fonction quadratique en utilisant la notation précédente mais en divisant le modèle en ses expressions linéaires et quadratiques : 
	\begin{equation*}
	\tilde{f}(x) = \alpha_{L}^{T}\xi_L(x) + \alpha_{Q}^{T}\xi_Q(x).
	\end{equation*}
	L'indice $L$ dénote les termes linéaires et d'ordre 0 de $\xi(x)$, au compte de $n+1$, soient les $n$ variables et le terme de degré 0, $\xi_0(x)=1$. L'indice $Q$ dénote les termes quadratiques au compte de $(n+1)(n+2)/2 - (n+1) = \frac{n(n+1)}{2}$. Ainsi, on peut réecrire la quadratique en trois termes, soient le terme constant, le terme des composantes linéaires et le terme des composantes quadratiques : 
	\begin{equation*}
	\tilde{f}(x) = c + g^T x + \frac{1}{2} x^T H x.
	\end{equation*}
	Avec $g \in \R^n$ et la matrice $H\in \R^{n\times n}$ la matrice Hessienne symétrique du modèle. Avec un problème sous déterminé où $p<q$, on choisira le modèle tel que :
	\begin{align*}
	&\underset{H \in \R^{n,n}}{\min}& &\norm{H}^2_H & &\\
	&\text{sujet à} & &c + g^T y^i + \frac{1}{2} (y^i)^T H (y^i) = f(y^i) & &i = 1,\dots, p.
	\end{align*}
	On entend ici minimiser l'influence des termes quadratiques pour diminuer l'amplitude du modèle entre les points de $Y$ utilisés. Par exemple, pour un problème de dimension $n$ avec $p\leq(n+1)$, on résoudera seulement la portion linéaire de $\alpha ^T \xi(y^i) = f(y^i)$ pour ainsi laisser $h_{i,j}=0, i,j=1\dots n$ et donc $\alpha_{Q}=0$.
             % Premier thème (Doctorat) ou "Détails de la Solution" (Maîtrise).
\Chapter{OPPORTUNISME ET ORDONNANCEMENT}\label{sec:Theme2}
Dans le contexte d'optimisation de boîte noire, chaque évaluation de la fonction objectif requiert un temps d'attente non-négligeable par l'utilisation de ressources informatiques; il est indispensable de veiller à ce que l'algorithme choisi soit conçu avec des spécificités réduisant le nombre d'appels à la boîte noire. Dans cette optique, les étapes de \POLL des algorithmes présentés précédemment sont revues.
\section{Stratégie Opportuniste}
Pour les algorithmes présentés à la section précédente, les analyses de convergence reposent sur une suite d'échecs à l'étape de \POLL, tandis que les étapes de \SEARCH sont d'avantage accessoires et simplement destinées à accélérer la convergence en pratique. Afin d'adapter les méthodes de recherche directe aux problèmes pouvant être traités comme des boîtes noires, il est primordial d'étudier chaque aspect pouvant entraîner une réduction d'évaluation de la fonction objectif dans la \POLL sans interférer avec ses propriétés. Ainsi, l'outil fourni est soutenu par une base théorique forte prouvant l'obtention d'une solution optimale~\cite{Torc97a,CoPr01a,AuDe2006,Kelley2011,KoLeTo03a} et est muni d'un processus soigneusement conçu pour éviter les évaluations couteuses. Les quelques définitions suivantes seront nécessaires à l'introduction de la stratégie principale qui reste à être formellement introduite.
\theoremstyle{definition}
\begin{definition}[Ensembles de recherche et de sonde]\label{def:3.1}
L'ensemble de recherche, désigné par $S^k$, est l'ensemble de points candidats pour l'étape de \SEARCH. L'ensemble de sonde, désigné $P^k$, est l'ensemble des points candidats pour l'étape de \POLL à l'itération $k$ d'un algorithme de recherche directe.
\end{definition} 
Par exemple, pour \CS, $S^k := \{\}$~\text{et}~$P^k :=\{x^k +\delta^ke_i ~\text{avec}~ i \in \{1,2,\dots,n\cup\N\}\}$. 
\begin{definition}[Diminution simple]
\label{def:3.2}Pour un algorithme de recherche directe, on dit qu'un candidat de $S^k$ entraîne une diminution simple si 
\begin{equation}
\label{eq:3.1}\exists~\tau~\in S^k~ \text{tel que}~ f(\tau) < f(x^k).
\end{equation}
Analoguement, on dit qu'un candidat dans $P^k$ entraîne une diminution suffisante si 
\begin{equation}
\label{eq:3.2}\exists~\tau~\in P^k ~\text{tel que} ~ f(\tau) < f(x^k).
\end{equation}
\end{definition}
La présence d'un candidat dans $P^k$ entraînant une diminution simple peut être un critère de succès pour une étape de \POLL. Ce n'est pas toujours le cas, tel que présenté précédemment dans \GSS, où le critère est de succès d'une \POLL requiert une diminution suffisante $\exists~\tau~\in P^k ~\text{tel que} ~ f(\tau) < f(x^k) + \rho$, où $\rho \geq 0$ est la différence minimale désirée entre les deux valeurs de la fonction objectif. Outre les diminutions simple et suffisante, on introduit un autre critère caractérisant un candidat.
\begin{definition}[Meilleure diminution]
	\label{def:3.3}Pour une étape de \POLL d'un algorithme de recherche directe, on dit qu'un candidat $\tau$ de $P^k$ entraîne la meilleure diminution de la fonction objective si 
	\begin{equation*}
	f(\tau) \leq f(p^k)~ \forall~ p^k \in P^k
	\end{equation*}
\end{definition}
Le cas traité ici est un l'obtention d'un optimum en utilisant un algorithme dans sa forme séquentielle. En pratique, il s'agit d'évaluer les points de façon séquentielle ou parallèle. Cependant, dans le cas d'optimisation de boîte noire, il est nécessaire de choisir si l'utilisation des ressources parallèles sont destinées à l'évaluation de la boîte noire, ou si elles sont dédiées à une version parallèle de l'algorithme~\cite{HoKoTo01a,AuDeLe08}. Il est intéressant de paralléliser un algorithme pour un ensemble de tâches indépendantes, par exemple dans l'optimisation de paramètres algorithmiques~\cite{AuDaOr13a}. Dans le cas où la boite noire est lourde en calculs et nécessite de grandes ressources en parallèle pour une évaluation unique, il est envisageable d'utiliser un algorithme dans sa version en séquentielle. L'utilisation d'un algorithme de recherche directe séquentiel signifie qu'au moment initial d'une étape de \POLL, $2n$ évaluations en série de la boîte noire sont possiblement à venir. Dans les algorithmes présentés, pour considérer la \POLL courante comme un succès, on n'exige pas la détermination d'un candidat entrainant la meilleure diminution au sens de la définition \ref{def:3.3}; elle exige la détermination d'un candidat apportant une diminution simple. Ainsi, pour le premier candidat satisfaisant Léquation (\ref{eq:3.2}), l'algorithme pourrait passer à l'étape suivante en considérant la présente comme un succès. Alternativement, on pourrait aussi continuer à évaluer la fonction objectif aux autres points de l'ensemble de sonde, de façon à identifier le candidat entraînant la meilleure diminution. Ces décisions algorithmiques sont le cœur de l'étude présente. 
\begin{definition}[Stratégie opportuniste (ou opportunisme)]
La stratégie opportuniste désigne l'arrêt prématuré de l'étape \SEARCH ou \POLL courante d'un algorithme de recherche directe dès la première obtention d'un point $\tau$ satisfaisant le critère de succès de l'algorithme, soit la diminution simple ou la diminution suffisante.
\end{definition}
En opposition à la stratégie opportuniste, on définit l'idée consistant à évaluer tous les points de l'ensemble $P^k$ ou $S^k$.
\begin{definition}[Sonde complète et recherche complète]
La sonde complète désigne l'évaluation de la fonction objectif à tous les points candidats de $P^k$ de l'étape de \POLL sans égard à l'identification encourue d'un candidat entrainant une diminution simple ou suffisante. 

La recherche complète désigne l'évaluation de la fonction objectif à tous les points candidats de $S^k$ de l'étape de \SEARCH sans égard à l'identification encourue d'un candidat entrainant une diminution simple ou suffisante. 
\end{definition}
Ces définitions concordent avec les mentions de l'opportunisme dans la littérature. Coope et Price~\cite{CoPr01a} sont les premiers à utiliser le terme d'opportunisme. De prime abord, ils identifient deux cadres de travail pour des algorithmes d'optimisation sans-contraintes dont les points sont limités à des maillages, soit les cadres A et B. Les deux cadres peuvent être résumés ainsi.
\begin{algorithm}[H]
	\caption{\textsf{Cadre algorithmique opportuniste(A) de Coope et Price}}
	\label{A}
	\begin{algorithmic}
	\STATE $f:\R^n\rightarrow \R$ la fonction objectif et $x^0$ le point de départ.
	\STATE \begin{tabularx}{440pt}{l X}1. & Déterminer un pas $\delta ^k$ et une base positive ordonnée $D^k:={d_1^k,d_2^k,\dots,d^k_l}$. Fixer $i \leftarrow 1$.\end{tabularx}
	\STATE \begin{tabularx}{440pt}{l X}2. & Déterminer une direction de descente $d_i^k\in D^k$ qui emmène une diminution simple de la fonction. Si une telle direction existe, mettre à jour la solution courante $x^k$. Sinon, essayer avec la prochaine direction dans la base positive ordonnée en incrémentant $i\leftarrow i+1$. Si aucun $i$ ne satisfait la condition, passer à l'étape 3.\end{tabularx}
	\STATE \begin{tabularx}{440pt}{l X}3. & Déterminer un ensemble de points finis avec une procédure arbitraire (\SEARCH). Si cette procédure produit une diminution simple de la fonction objective, mettre à jour la solution. Retour à 1.\end{tabularx}
	\end{algorithmic}
\end{algorithm}
\begin{algorithm}[H]
	\caption{\textsf{Cadre algorithmique non-opportuniste (B) de Coope et Price}}
	\label{A}
\begin{algorithmic}
		\STATE $f:\R^n\rightarrow \R$ la fonction objectif et $x^0$ Le point de départ 
		\STATE  \begin{tabularx}{440pt}{l X}
			1. & Déterminer un pas $\delta ^k$ et une base positive ordonnée $D^k$. $i \leftarrow 1$.
		\end{tabularx}
	\STATE \begin{tabularx}{440pt}{l X}
		2. & Déterminer la direction menant à la meilleure descente $d_i^k$ dans $D^k$. Faire une recherche linéaire car cette direction mène à un succès. Mettre $x^k$ à jour et recommencer. Si aucune direction de descente est trouvée, un minimum sur le maillage est atteint.
	\end{tabularx}
		\STATE \begin{tabularx}{440pt}{l X}3. & Déterminer un ensemble de points finis avec une procédure arbitraire (\SEARCH). Si cette procédure produit une diminution simple, mettre à jour la solution. Retour à 1.
		\end{tabularx}
\end{algorithmic}
\end{algorithm}
Les auteurs stipulent que le cadre algorithmique B est une adaptation du A et enchaînent avec la démonstration que la procédure répétée de A mène ultimement à un la détermination d'un point stationnaire de la fonction $f$. Ils amènent le point que l'algorithme B est beaucoup plus restreint que l'algorithme A. Par contre, c'est la routine primitive de A qui assure la convergence, malgré que le candidat entraînant la meilleure diminution n'est pas obligatoirement identifié. C'est ici qu'ils mentionnent l'opportunisme associé à ce concept algorithmique, la première instance dans littérature au meilleur de notre connaissance.
\begin{quote}
	<<Pour le cadre algorithmique A, la convergence peut être démontrée seulement pour une sous-séquences de minimum locaux du maillage. Ceci est dû au fait que la recherche effectuée à l'étape 2 est opportuniste; le premier membre rencontré dans $D$ qui donne une descente (diminution simple) mène à un nouvel itéré.>>
\end{quote}
La concept de l'opportunisme, souvent désigné autrement dans la littérature, impacte l'analyse de convergence. Dans~\cite{Torc97a}, Torczon introduit les mouvements exploratoires, sur lesquels l'analyse de convergence de \GPS, et par extension \CS, reposent. Deux résultats sont identifiés. En premier lieu, on statue que la convergence globale est assurée pour une méthode de \GPS sur une fonction objectif $f$ continue différentiable et sur un ensemble avoisinant un ensemble compacte. La conclusion étant que sous ces conditions, une série infinie d'évaluations mène à un point ou la norme du gradient est nulle.
\begin{equation}
	\label{eq:3.4}\liminf\limits_{x\rightarrow \infty}\norm{\nabla f(x_k)} = 0
\end{equation}
Ce résultat peut être renforcé par des hypothèses plus strictes sur la norme des colonnes de la matrice génératrice $Z$, sur le paramètre d'ajustement du maillage $\tau$ et, plus important encore, les hypothèses de mouvements exploratoires. Initiallement, ces hypothèses se limitaient à 
\begin{itemize}
	\item La direction $d$ est choisie $D=GZ$ et la longueur est contrôlée par $\delta^k$ 
	\item Si un des points existants dans $P^k$ entraîne une diminution simple, alors un mouvement exploratoire produira un pas qui entrainera une diminution simple.
\end{itemize}
Le résultat de convergence s'atteint en conservant la première hypothèse et en remplaçant la deuxième par la suivante : 
\begin{itemize}
	\item Si un des points existants dans $P^k$ entraîne une diminution simple, alors un mouvement exploratoire produira un pas qui entrainera la meilleure descente. 
\end{itemize}
Le résultat de convergence renforcé stipule :
\begin{equation}
	\label{eq:3.5}\lim\limits_{x\rightarrow \infty}\norm{\nabla f(x_k)} = 0.
\end{equation}
L'hypothèse sur les mouvements exploratoires est uniquement applicable si tous les points de $P^k$ sont évalués. Il en découle que la convergence est influencée par l'opportunisme. L'utilisation de la stratégie opportuniste contredit l'hypothèse nécessaire pour la convergence forte, ainsi une implémentation de \GPS opportuniste assure seulement une convergence dans le sens de la limite inférieure. Toujours dans~\cite{Torc97a}, l'auteur adapte \CS au cadre énoncé pour une méthode de \GPS, entraînant ainsi le résultat de convergence applicable à être applicable à \CS. Une situation semblable est répétée dans~\cite{KoLeTo03a} pour \GSS, soit que la convergence plus forte de l'équation (\ref{eq:3.5}) est atteignable avec la prémisse que $x^{k+1}$ soit le candidat amenant la meilleure descente seulement. 
Les algorithmes présentés dans dans la section 2 ne vont pas dans les détails de leurs implémentations et la présence de l'opportunisme y est laissée floue. Conn, Scheinberg et Vincente~\cite{CoScVibook} évoque explicitement la présence de l'opportunisme dans les cadres algorithmiques de méthodes de recherche directe directionnelles qu'ils utilisent.
\section{Ordonnancement}
L'utilisation de l'opportunisme entraîne que l'ordre des points dans l'ensemble de sonde $P^k$ prend soudainement un rôle important. Dans l'optique de réduire le nombre d'appels à la boîte noire, on s'intéresse ici à la possibilité de réordonner $P^k$ de façon à évaluer les candidats possiblement intéressants en premier lieu. Dans cet ordre d'esprit, Custodio et Vincente~\cite{CuVi07} tentent de déterminer des indicateurs de descente, c'est à dire une direction avec un potentiel de descente intéressant identifié à l'aide des informations obtenus à présent sur le problème. 
\begin{definition}[Ordonnancement]
	\label{def_ordo}L'ordonnancement réfère à la permutation des directions, soient des colonnes de $P^k$ ou $S^k$, suivant une stratégie donnée. Une stratégie guidant un ordonnancement est désignée comme stratégie d'ordonnancement.
\end{definition}
 Dans leur étude, Custodio et Vincente identifient les dérivées du simplexe, calculables à l'aide des évaluations précédentes, compte tenu de certaines contraintes sur la géométrie de l'ensemble de points choisis~\cite{CoScVi2006}. Ils utilisent la distance angulaire minimale entre $-\nabla_{s}f(x)$ et $d_i$ comme stratégie d'ordonnancement. Les résultats numériques montrent que, sur une version opportuniste de \CS, cet ordre réfléchi des points dans $P^k$ est grandement bénéfique lorsque comparé à un ordre statique, surtout si combiné à d'autre stratagèmes identifiés par les auteurs. De ces stratégies on nomme l'utilisation des informations obtenues sur chaque point (en opposition à utiliser seulement les points garantissant une diminution simple) pour assurer la répartition de l'ensemble de points ainsi que l'expansion de la longueur du pas limitée à une série de succès consécutifs issues de la même direction de succès, stratégie proposée dans~\cite{HoKoTo01a}. Custodio, Dennis et Vincente~\cite{CuDeVi08} réutilisent la stratégie dans le contexte d'optimisation non-lisse et observent toujours une amélioration en comparant avec un ordre arbitraire restant inchangé au cours de l'optimisation. Abramson, Audet et Dennis~\cite{AbAuDe04a} montrent quant à eux qu'il est possible d'élaguer l'ensemble $P^k$ de façon à n'avoir qu'une seule évaluation par étape de \POLL fructueuse en utilisant les informations sur les signes des dérivés. Lorsque Audet et Dennis introduisent \MADS~\cite{AuDe2006}, ils spécifient que l'ordre des directions dans leurs ensembles $P^k$ sont aléatoirement déterminés.  
   
Définition de précéder $\prec$

\subsection{Ordonnancement lexicographique}


\subsection{Ordonnancement en fonction du dernier succès} : =
\begin{equation}
\frac{d_{i,k}\cdot d_{i,k-1}}{||d_{i,k-1}||||{d_{i,k}||}} > \frac{d_{j,k}\cdot d_{k-1}}{||d_{k-1}||||{d_{j,k}||}}  \implies d_{i,k} \prec d_{j,k}  
\end{equation}


\subsection{Ordonnancement en fonction du modèle} : 
\begin{equation}
\tilde{f}(p_i) < \tilde{f}(p_j) \implies p_i \prec p_j
\end{equation}


\subsection{Ordonnancement aléatoire}
Hasard.\\

\subsection{Ordonnancement omniscient}
Stratégie pour tester "borne supérieure"
\begin{equation}
\tilde{f}(p_i) < \tilde{f}(p_j) \implies p_i \prec p_j
\end{equation}

\subsection{Ordonnancement négatif-omniscient}
Stratégie pour tester "borne inférieure"
\begin{equation}
\tilde{f}(p_i) < \tilde{f}(p_j) \implies p_i \prec p_j
\end{equation}

\section{Mise en place de l'opportunisme}
En forçant l'évaluation séquentielle de la fonction sur les candidats de $P^k$ et $S^k$, l'opportunisme dicte la forme à prendre pour les algorithmes. Ainsi, leurs cadres proposés dans la littérature sont raffermis et limités à des formes séquentielles. Cependant, l'application de la stratégie est limitée à certains algorithmes dont la forme permet son instauration. Certains algorithmes tels que l'algorithme de Nelder Mead~\cite{NeMe65a} et les algorithmes de régions de confiance~\cite{CoScVibook} en optimisation sans-dérivées possèdent un seul candidat par itération de l'algorithme. L'ordonnancement est donc incompatible avec certains algorithmes, ce qui justifie la tâche d'identifier préalablement un ensemble d'algorithmes compatibles. Le critère de compatible principal consiste à l'existence d'une étape ou une liste de points doit être évaluée, soient les étapes de \SEARCH et de \POLL ainsi que dans leur équivalent pour \imfil. La section qui suit détaillera la modification des étapes de \POLL pour les algorithmes de recherche directe identifiés préalablement. L'opportunisme pour les étapes de \SEARCH, quoique figurant comme paramètre pour certaines implémentations~\cite{Le09a}, est laissée de la côté pour la suite du présent travail, compte tenu de l'extrême versatilité de l'étape de \SEARCH.
\subsection{\CS, \GPS, \GSS et \MADS}
À des fins de généralité, les ensembles des directions de sonde utilisés dans \CS et dans \MADS, soient l'ensemble des directions élémentaires et $\D^k_\Delta$ seront substitués par $D^k$. 
\begin{algorithm}[H]
	\caption{\textsf{\POLL opportuniste}}
	\label{alg1}
	\begin{algorithmic}
		\STATE Avec $f:\R^n \rightarrowtail \R$ la fonction objectif et $x^0$ le point de départ
		\STATE 1. \POLL
		\bindent
		\STATE Ordonner $P^k$ selon la stratégie voulue.
		\FOR { $t \in P^k$}
		\IF {$f(t) < f(x^{k}) $}
		\STATE $x^{k+1} \leftarrow t$
		\STATE Mise à jour de $\delta^k$ et $\Delta^k$ selon un succès de l'étape
		\STATE Poursuivre à la Terminaison.
		\ENDIF
		\STATE $x^{k+1} \leftarrow x^{k}$ et $\delta^{k+1} \leftarrow \tau\delta^k$
		\STATE Mise à jour de $\delta^k$ et $\Delta^k$ selon un échec de l'étape
		\ENDFOR
		\eindent
	\end{algorithmic}
\end{algorithm}
\subsection{\imfil}
L'utilisation de l'opportunisme dans \imfil nécessite une adaptation partielle de la méthode. Premièrement, la notion de succès pour \imfil est différente de celle établie pour les autres algorithmes identifiés. On dira que la sonde du gradient comporte un succès si il existe un point $F_i$ dans $\{F\}$ pour lequel la fonction objective est simplement diminuée. Pour respecter \ref{def_ordo} sans compromis, on devrait terminer la sonde du gradient au premier succès. Cependant, \imfil enchaîne cette sonde peu raffinée avec une descente du gradient, qui constitue apporte substance à l'algorithme et lui confère un intérêt. L'interruption trop précoce de la sonde du gradient empêchera l'obtention de l'information nécessaires pour la détermination d'une direction de descente cohérente, puisque celles-ci sont déterminées à l'aide de différences finies obtenues avec les points de $F$. Par exemple, à l'obtention d'un succès à une première itération, l'algorithme effectuera une descente du gradient avec $n-1$ directions nulles. On désignera cette méthode comme \textsf{\imfil-op} pour \imfil avec opportunisme pur. On devine que cette stratégie nuira au bon fonctionnement pratique de la méthode.
\begin{algorithm}[H]
 	\caption{\textsf{Sonde du gradient \imfil-op}}
 	\label{alg5}
 	\begin{algorithmic}
 		\STATE Avec $f:\R^n \rightarrow \R$ la fonction objectif et $x^0$ le point de départ
 		\STATE Effectuer la sonde du gradient
 		\FOR {$i=1,\dots,k$}
 		\STATE $F_i = f(x^k + h^k v_i), v \in V$
 		\IF {$F_i < x^k$}
 		\STATE $x_{\min} \leftarrow F_i$
 		\STATE Poursuivre au calcul de la direction de descente
 		\ENDIF
 		\ENDFOR
 	\end{algorithmic}
\end{algorithm}

On désignera cette méthode comme \textsf{\imfil-od} pour \imfil avec opportunisme décalé. 
  
\begin{algorithm}[H]
	\caption{\textsf{Sonde du gradient \imfil-od}}
	\label{alg5}
	\begin{algorithmic}
		\STATE Avec $f:\R^n \rightarrowtail \R$ la fonction objectif et $x^0$ le point de départ
		\STATE Effectuer la sonde du gradient
		\FOR {$i=1,\dots,n$}
		\STATE $F_i = f(x^k + h^k v_i), v \in V$
		\ENDFOR
		\IF {$f(\arg \min(F)) < f(x_k)$}
		\STATE $x_{\min} \leftarrow\arg \min(F)$
		\STATE Poursuivre au calcul de la direction de descente
		\ENDIF
		\FOR {$i=n+1,\dots,k$}
		\STATE $F_i = f(x^k + h^k v_i), v \in V$
		\IF {$F_i < x^k$}
		\STATE $x_{\min} \leftarrow F_i$
		\STATE Poursuivre au calcul de la direction de descente
		\ENDIF
		\ENDFOR
	\end{algorithmic}
\end{algorithm}  
   
La méthode \textsf{\imfil-od} ===> erreur $O(h)$ plutôt que $O(h^2)$ dans le calcul de la dérivée.
\section{Modèles Quadratriques pour l'ordonnancement}
L'ordonnancement guidé par l'utilisation modèles quadratiques élaborés dans~\cite{CoLed2011} est possible avec les algorithmes présent dans NOMAD, tels que CS, GPS et MADS. Pour obtenir un modèle, on considère la base naturelle de l'espace des polynômes de degré deux et moins. 
\begin{equation*}
\xi (x)=(\xi_0(x),\xi_1(x),...,\xi_q(x))^T = \left(1,x_1,x_2,...,x_n,\frac{x_1^2}{2},\frac{x_2^2}{2},...,\frac{x_n^2}{2},x_1 x_2, x_1 x_3,...,x_{n-1},x_{n}\right)^T
\end{equation*}
Cette base possède $q+1 = (n+1)(n+2)/2$ éléments. Le modèle $m_f$ de la fonction $f$ est tel que $\tilde{f}(x)=\alpha^T\xi(x)$, $\alpha \in \R^{q+1}$. Pour obtenir ce modèle, un ensemble de point $Y=\{y^0,y^1,...,y^p\}$ ayant $p+1$ éléments est nécessaire. On cherche alors a minimiser la différence entre les valeurs de la boîte noire évaluées aux points de $Y$ et celles du modèle. Au long de l'exécution présente ou de celles passées enregistrées dans les caches, les algorithmes évalueront la boîte noire à différents points, parmis lesquels pouront être choisis les $p = q$ points nécessaires à l'élaboration du modèle. Les points devront satisfaire la propriété qui valide le modèle : 
\begin{gather*}
B(\xi,Y)\alpha = f(Y)\\
f(y)=(f(y^0,f(y^1),...,f(y^p))^T\\
B(\xi,Y) = 
\begin{bmatrix}
\xi_0(y^0) & \xi_1(y^0) & \dots & \xi_q(y^0)\\
\xi_0(y^1) & \xi_1(y^1) & \dots & \xi_q(y^1)\\
\vdots & \vdots & \vdots & \vdots\\
\xi_0(y^p) & \xi_1(y^p) & \dots & \xi_q(y^p)\\
\end{bmatrix}_.
\end{gather*}
Ce système peut être résolu seulement si $p=q$ et si la matrice est de rang pleine. Dans le cas où $p\geq q$, on tentera de résoudre le problème de minimisation suivant : 
\begin{equation*}
\begin{aligned}
& \underset{\alpha \in \R}{\text{min}}
& & \norm{B(\xi,Y)\alpha -f(Y)}^2
\end{aligned}.
\end{equation*}
Dans le cas où $p<q$, on minimisera le même problême mais en régularisant le problème à l'aide d'une interpolation dans le sense de la norme de Frobenius minimale~\cite{MoWi2009,CuRoVi10}. La norme de Frobenius pour une matrice $A$ est définie par : 
\begin{equation*}
\norm{A}_F = \sqrt{\overset{m}{\underset{i=1}{\sum}} \overset{n}{\underset{j=1}{\sum}}|a_{i,j}|}.
\end{equation*}
Il est possible de réarranger $\tilde{f}(x)$ de façon à l'illustrer comme une fonction quadratique en utilisant la notation précédente mais en divisant le modèle en ses expressions linéaires et quadratiques : 
\begin{equation*}
\tilde{f}(x) = \alpha_{L}^{T}\xi_L(x) + \alpha_{Q}^{T}\xi_Q(x).
\end{equation*}
L'indice $L$ dénote les termes linéaires et d'ordre 0 de $\xi(x)$, au compte de $n+1$, soient les $n$ variables et le terme de degré 0, $\xi_0(x)=1$. L'indice $Q$ dénote les termes quadratiques au compte de $(n+1)(n+2)/2 - (n+1) = \frac{n(n+1)}{2}$. Ainsi, on peut réecrire la quadratique en trois termes, soient le terme constant, le terme des composantes linéaires et le terme des composantes quadratiques : 
\begin{equation*}
\tilde{f}(x) = c + g^T x + \frac{1}{2} x^T H x.
\end{equation*}
Avec $g \in \R^n$ et la matrice $H\in \R^{n,n}$ la matrice Hessienne symétrique du modèle. Avec un problème sous déterminé où $p<q$, on choisira le modèle tel que :
\begin{align*}
&\underset{H \in \R^{n,n}}{\text{min}}& &\norm{H}^2_H & &\\
&\text{sujet à} & &c + g^T y^i + \frac{1}{2} (y^i)^T H (y^i) = f(y^i) & &i = 1,\dots, p.
\end{align*}
On entend ici minimiser l'influence des termes quadratiques pour diminuer l'amplitude du modèle entre les points de $Y$ utilisés. Par exemple, pour un problème de dimension $n$ avec $p\leq(n+1)$, on résoudera seulement la portion linéaire de $\alpha ^T \xi(y^i) = f(y^i)$ pour ainsi laisser $h_{i,j}=0, i,j=1\dots n$ et donc $\alpha_{Q}=0$.             % Second thème (Doctorat) ou "Résultats théoriques et expérimentaux" (Maîtrise).
\Chapter{RESULTATS NUMÉRIQUES}\label{sec:Theme3}
\section{Outils de comparaison}\label{sec:out}
Les premières recherches visant à comparer les performances de solveurs sur un problème unique ou une série de problème utilisaient des métriques facilement quantifiables telles que le nombre d'appels à la fonction~\cite{BoCoGoTo1995} ainsi que le temps de résolution en secondes~\cite{Mi99}. Cependant, ces indicateurs sont moins appropriés en situation de résolution d'un grand nombre de problèmes. Puisque ces métriques sont spécifiques à un problème, le banc d'essai qui incorpore chacune d'entre-elles prends des proportions énormes. Dans cet ordre d'esprit, certains outils ont été développés pour analyser une série de problèmes test, avec des dimensions et des spécificités différentes, sans qu'il en devienne laborieux d'en tirer des conclusions.
\subsection{Test et graphe de convergence}\label{sec:con}
Pour comparer la performance d'un ensemble d'algorithmes sur un seul problème, on aura recours au graphe de convergence. On y trace la courbe de la meilleure valeur de la fonction objectif obtenue en fonction du nombre d'appel à la fonction objectifs.

En optimisation sans-dérivées, contrairement à  en optimisation classique, rien n'assure qu'un solveur arrivera à l'optimum global, ou encore qu'un optimum soit atteint dans le budget alloué. De plus, en situation de résolution de boîte noire, il est impossible de connaître l'optimum théorique du problème, et ce si celui-ci existe. On aimerait tout de même quantifier la performance d'un solveur.  À cette fin, il est entendu dans la littérature de faire appel à un test de convergence qui mesure la capacité à l'algorithme à s'approcher de la meilleure solution obtenue par l'ensemble des algorithmes à l'étude, soit 
\begin{equation*}
f(x^0) - f(x) \geq (1-\tau)(f(x^0)-f^*)
\end{equation*}
issu de~\cite{MaNo02a} avec $x^0$ le point initial, $x$ le point courant, $f$ la fonction objectif, $\tau\geq0$ un seuil de tolérance et $f^*$ la meilleure solution courante trouvée par un des solveurs comparés sur ce problème. Le test de convergence peut être utilisé pour savoir si un solveur atteint au moins une solution proche de la meilleure en utilisant un $\tau$ grossier, par exemple $\tau = 0.1$, ou encore pour savoir si le solveur est capable d'obtenir $f^*$ en utilisant $\tau =0$. Il est possible que le test ne soit jamais réussi si on observe la résolution avec un budget d'évaluation fixe, ce qui sera nécessaire pour des boîtes noires bruitées, pour lesquelles la convergence de l'algorithme n'est pas assurée.

En situation de barrière progressive, le test de convergence doit être adapté pour tenir compte de l'impertinence de $f(x)$ pour $x\notin \Omega$. Afin de comparer seulement des points réalisables, on utilisera le test
\begin{equation*}
\bar{f}_{\textsf{fea}} - f(x_e) \geq(1-\tau)(\bar{f}_{\textsf{fea}}-f^*).
\end{equation*}
issu de~\cite{AuTr17}. $\bar{f}_{\textsf{fea}}$ y représente la valeur moyenne des premiers points réalisables obtenus dans les résolutions des différentes instances du même problème et $f(x_e)$ le meilleure valeur de la fonction objectif obtenue à un point réalisable.
\subsection{Profil de performance}\label{sec:pdp}
Un outil établi pour la comparaison de solveurs d'optimisation classique est le profil de performance~\cite{DoMo02}. La performance est donnée par une mesure $t_{p,s}$, qui peut être par exemple le nombre d'évaluations nécessaire pour l'obtention du minimal global, pour une paire de $p$ et $s$, un problème et un solveur. En optimisation sans-dérivées, on choisira le nombre d'itérations nécessaire à la satisfaction du test de convergence énoncé précédemment. Afin de comparer les solveurs sur chaque problème, on défini le ratio de comparaison suivant
\begin{equation*}
r_{p,s}=\frac{t_{p,s}}{\min\{t_{p,s} : s\  \epsilon \  S\}}
\end{equation*}
On peut alors définir le profil de performance d'un solveur sur un ensemble de problème $P$ ainsi
\begin{equation*}
\rho_s(\alpha)=\frac{1}{|P|}\text{size}\{p\ \epsilon \ P : r_{p,s} \leq \alpha \}
\end{equation*}
Pour un $\alpha$, soit un indice de performance tel que le ratio de comparaison $r_{p,s}$, on peut savoir la proportion de problème que chaque solveur a su résoudre en se rapprochant à un facteur de $(1-\tau)$ de la meilleure solution trouvée par l'ensemble des solveurs. Le test de convergence varie selon la nature des problèmes à résoudre. On peut utiliser un profil de performance pour des outils d'optimisation avec dérivés avec un $t_{p,s}$ qui agit sur les informations de premier ordre de la fonction, tel que le nombre d'évaluations pour atteindre $x$ tel que $f'(x)=0$. Ainsi, $\rho_s(\tau)$ revient à être la probabilité que le ratio de performance $r_{p,s}$ du solveur $s$ soit à un facteur $\tau$ du meilleur ratio possible. La lacune principale des profils de performance consiste en l'incapacité du ratio utilisé de prendre en compte la dimension du problème.

Il est important de noter que les profils  de performance ont une faille tel qu'élaboré dans~\cite{GoSc2016}. Les auteurs concluent que l'utilisation de $f^*$ dans le test de convergence implique que la simple observation des courbes sur un graphe n'est pas suffisante pour classes les algorithmes. Puisque la comparaison se fait avec le nombre d'itérations minimal observé, il se peut que, lorsqu'on compare entre-eux deux solveurs qui n'ont pas performés en tant que meilleur, l'ordre apparent ne soit pas celui qu'on observerait si on éliminait du graphe les données du meilleur. On se doit alors d'utiliser un autre outil pour mesurer les performances des solveurs sur un ensemble de données.
\subsection{Profil de données}\label{sec:pdd}
Moré et Wild proposent~\cite{MoWi2009} une définition d'un profil de donnée, 
\begin{equation*}
d_s(\alpha)=\frac{1}{|P|}\text{size}\{p\ \epsilon \ P : \frac{t_{p,s}}{n_p+1} \leq \alpha \}\\
\end{equation*}
avec $t_p$ le nombre d'évaluations pour atteindre la convergence nécessité par le solveur $s$ sur le problème $p$. Le facteur $n_p+1$ est le nombre d'évaluations nécessaire pour le calcul de l'estimation finie d'un gradient pour un problème de taille $n_p$. Il est alors possible de mesurer la performance relative des solveurs comme une fonction du budget d'évaluations, sans la comparaison que le ratio de performance imposait. Ainsi, pour un $\alpha$, soit une quantité de $n+1$ évaluations de la fonction objectif, on peut savoir la proportion de problèmes que chaque solveur a su résoudre en se rapprochant à un facteur de $(1-\tau)$ de la meilleure solution trouvée par l'ensemble des solveurs. L'avantage des profils de données sur les profils de performance est qu'il est possible de quantifier la performance du meilleur algorithme, compte tenu que celui-ci n'est plus comparé à lui-même. La faille évoquée par Gould et Scott~\cite{GoSc2016} n'est pas applicable aux profils de données puisqu'ils ne nécessitent
\section{Problèmes test}\label{sec:pro}
La comparaison des stratégies se fait sur plusieurs ensembles de problèmes tests. En premier lieu, un ensemble de problèmes non-linéaires sans contraintes et sans bornes est décrit afin de caractériser le comportement des algorithmes opportunistes sur un large ensemble de problèmes. En deuxième lieu, un échantillon de problèmes avec des contraintes relaxables et quantifiables est décrit afin d'observer la performance des différentes versions de la stratégies opportuniste en situation de barrière progressive. Enfin, on décrit en détails des cas de problèmes d'ingénierie connus comme pouvant être résolus avec des outils d'optimisation sans-dérivées dans le but de déterminer si la stratégie opportuniste a le potentiel d'influer la résolution d'une boîte noire d'expression non analytique.
\subsection{Ensemble de problèmes Moré-Wild}\label{sec:mw}
L'échantillon de problèmes analytiques caractéristiques $\Pset$ utilisé est celui proposé par Moré et Wild~\cite{MoWi2009}. Cet ensemble de problèmes est fréquemment réutilisé dans la communauté~\cite{VaVi07,CuRoVi10,ZhCoSc10,CoLed2011,AdAuBeYa2014,,AuLedTr2014}, ce qui justifie son utilisation comme base de problème analytique pour la comparaison d'algorithme ou de stratégies algorithmiques. Il s'agit d'une banque de problème sans contraintes et sans bornes.

L'ensemble de problèmes est décliné de 22 fonctions non linéaires de moindres carrés tirées de la collection $\mathrm{CUTEr}$~\cite{GoOrTo03}. Ces 22 fonctions sont remaniées pour donner un ensemble de 53 problèmes. L'indice $k_p$ pour un problème $p \in \mathcal{P}$ fait référence à la fonction de base issue de $\mathrm{CUTEr}$ utilisée pour ce problème. 

Chaque problème possède trois autres paramètres en plus de l'indice $k_p$, soient $n_p$ pour la dimension du problème, $m_p$ le nombre de composantes du problème et le paramètre binaire $s_p$, indiquant si une homothétie de facteur 10 est activé ou non sur le point de départ $x_0$. L'ensemble $\Pset$ contient 53 problèmes pour lesquels existe un vecteur $ (k_p,n_p,m_p,s_p) $ unique. Aucune fonction sous-jacente n'est surreprésentée puisque au plus six problèmes possèdent le même $k_p$. Les bornes sur les paramètres sont 
\begin{equation*}
1 \leq k_p \leq 22,\ \ 2 \leq n_p \leq 12,\ \ 2\leq m_p\leq 65,\ \ s_p \in \{0,1\},\ \ p=1,\dots,53
\end{equation*}
avec $n_p \leq m_p$.

La structure par morceaux des problèmes de $\Pset$ permet de dériver l'ensemble en d'autres classes de problèmes. On entend ainsi permettre la comparaison d'algorithmes ou de stratégies algorithmiques sur différentes classes de problèmes dont les caractéristiques peuvent être représentés dans des problèmes de boîtes noires. Les classes utilisées dans~\cite{MoWi2009} qui sont réutilisées ici sont : les problèmes lisses, les problèmes lisses définis par partie, les problèmes à bruit stochastique et les problèmes à bruit déterministe.  \textbf{Les problèmes lisses} sont formés tels que : 
\begin{gather*}
f(x):=\sum_{i=1}^{m}{\left(f_i(x)\right)}^2.
\end{gather*}
\textbf{Les problèmes non-différentiables} sont obtenus à l'aide d'une transformation apportée à l'expression des problèmes lisses telle que : 
\begin{gather*}
f(x):=\sum_{i=1}^{m}{|f_i(x)|}.
\end{gather*}
Ainsi, l'expression du gradient $\nabla f_i(x)$ est inexistante lorsque $f_i(x)=0$. Afin d'assurer que chaque déclinaison de problème possède un minimum global, pour les fonctions issues d'un ensemble de problèmes sous-jacent $k_p \in\{ 8,9,13,16,17,18\}$, la fonction est redéfinie dans l'orthant positif tel que
\begin{gather*}
f(x):=\sum_{i=1}^{m}{|f_i(\max(x,0))|}.
\end{gather*}
\textbf{Les problèmes à bruit déterministe} sont obtenus à l'aide d'une transformation apportée à l'expression des problèmes lisses. On multiplie la fonction $f(x)$ par le terme $(1+\epsilon_{f}\theta(x))$ qui induit une oscillation 
\begin{gather*}
f(x):=(1+\epsilon_{f}\theta(x))\sum_{k=1}^{m}{f_{i}(x)^{2}}.
\end{gather*}
Le terme $\theta(x)$ est issu d'une composition d'une fonction trigonométrique $\theta_0(x)$ avec un polynôme de Chebyshev de degré 3 tel que $T_{3}(\alpha) = \alpha(4\alpha^{2}-3)$
\begin{gather*}
\theta_0(x)=0.9\sin(100||x||_{1})\cos(100||x||_\infty)+0.1\cos(||x||_2) \\
\theta(x) = T_3(\theta_0(x)) = T_{3}(0.9\sin(100||x||_{1})\cos(100||x||_\infty)+0.1\cos(||x||_2)).
\end{gather*} 
Cette composition élimine la périodicité de $\theta_{0}$. La formulation est aussi composée du terme $\epsilon_f$, le niveau du bruit relatif, qui est fixé à $\epsilon_f = 10^{-3}$.  

\textbf{Les problèmes à bruit stochastique} sont obtenus avec une transformation apportée à chaque composante de la sommation 
\begin{gather*}
f(x):=\sum_{k=1}^{m}{{((1+\epsilon_f u_i)f_{i}(x))}^{2}}.
\end{gather*}
Le scalaire $u_i$ est issu du vecteur $u=(u_1,u_2,\dots,u_m)$ dont chaque valeur $-0.5<u_i<0.5$ est déterminée aléatoirement. La formulation est aussi composée du terme de bruit relatif fixé à $\epsilon_f = 10^{-3}$.
\subsection{Problèmes analytiques contraints}\label{sec:pac}
Les problèmes Moré-Wild n'ont pas de contraintes ni de bornes.  L'impact observé de l'opportunisme et de l'ordonnancement sur ces problèmes n'est pas nécessairement transférable en optimisation sous contraintes, compte tenu des différences que comporte la sonde lorsqu'on se trouve un situation de barrière active. Un ensemble de problèmes contraints issu de ~\cite{CoLed2011,AuTr17} est mis en place pour former un banc d'essai. Il s'agit de problèmes analytiques recueillis dans la littérature comportant des contraintes relaxables de type \textsf{QR*K}. Le nombre $n$ de variables varie de 2 à 20 et le nombre $m$ de contraintes de 2 à 11.
\begin{table}[h]
\centering
\begin{tabular}{||c r r r r||c r r r r||}
\hline
\# & Nom & Source & $n$ & $m$ & \# & Nom & Source & $n$ & $m$\\
\hhline{||=====||=====||}
1 & CHENWANG\_F2 & \cite{ChWa2010} & 8 & 6 & 10 & MAD6 & \cite{HoSc1981} & 5 & 7\\ 
2 & CHENWANG\_F3 & \cite{ChWa2010} & 10 & 8 & 11 & MEZMONTES & \cite{Montes2009} & 2 & 2\\ 
3 & CRESCENT & \cite{AuDe09a} & 10 & 2 & 12 & OPTENG\_RBF & \cite{KiArYa2011} & 3 & 4\\ 
4 & DISK & \cite{AuDe09a} & 10 & 1 & 13 & PENTAGON & \cite{LuVl00} & 6 & 15\\ 
5 & G2\_10 & \cite{AuDeLe08} & 10 & 2 & 14 & PIGACHE & \cite{PiMeNo2007} & 4 & 11\\ 
6 & G2\_20 & \cite{AuDeLe08} & 20 & 2 & 15 & SNAKE & \cite{HoSc1981} & 2 & 2\\ 
7 & HS19 & \cite{HoSc1981} &  2 & 2 & 16 & SPRING & \cite{AuDe09a} & 3 & 4\\ 
8 & HS83 & \cite{LuVl00} & 5 & 6 & 17 & TAOWANG\_F2 & \cite{TaWa08} & 7 & 4\\ 
9 & HS114 & \cite{HoSc1981} & 9 & 6 & 18 & ZHAOWANG\_F5 & \cite{ChWa2010b} & 13 & 9\\  
\hline
\end{tabular}
\caption{Description de l'ensemble de problèmes contraints}
\end{table}
\subsection{STYRENE}\label{sec:sty}
Les méthodes de recherche directe sont appliquées à la résolution d'un problème issue de l'ingénierie de procédé proposé par Audet et al.~\cite{AuBeLe08}. Il s'agit d'une simulation de la production de styrène, un composé organique utilisé dans la fabrication de plastique, particulièrement le polystyrène. Cette simulation comprends quatres étapes principales.
\begin{enumerate}[label=\roman*.]
\item La préparation des réactifs par pressurisation et par chauffage précède la réaction chimique. Cette étape est influencée par la pression à la sortie de la pompe $x_1$, en atm, et la température à la sortie du réchauffeur $x_2$, en K, qui constituent des variables du problème. Il est à noté que les entrées dans le système proviennent d'une source externe et d'un retour de matière recyclé provenant de la sortie du procédé chimique. 
\item La réaction chimique prenant place dans le réacteur, faisant intervenir $x_3$, en m, la longueur du réacteur.
\item La récupération du styrène par la distillation des produits de la réaction. Les produits sont initialement refroidis à une certaine température $x_4$, en K. Les produits sont dégazés et les gaz indésirables sont ensuite brulés et éjectés par la cheminée, dont la fraction d'air excédentaire $x_5$  pour la combustion figure comme variable d'optimisation. Les produits refroidis et dégazés vont ensuite au séparateur à styrène, dans lequel a lieu la première distillation. Le séparateur nécessite deux composantes, soit la lourde et la légère. La fraction $x_6$ de composante légère utilisée dans le séparateur constitue une variable du problème. À la sortie du séparateur, les résidus (ethylbenzène) peuvent être réutilisés comme réactifs. La fraction des résidus $x_7$ qui retourne à la première étape est une variable nécessaire pour la simulation.
\item La portion non-recyclée des résidus est ensuite distillée à son tour pour en retirer le benzène des éléments réactifs restants, lesquels sont aussi retourné à la première étape. La fraction $x_8$ de composante légère utilisée dans le séparateur à benzène constitue une variable du problème.
\end{enumerate}
Avec $x = (x_1,x_2,x_3,x_4,x_5,x_6,x_7,x_8)$ le vecteur de variables, le simulateur peut estimer la valeur nette estimée du projet de production de styrène, en agrégeant des facteurs comme les coûts d'opérations, le profit issus des ventes du styrène produit, les investissements nécessaires pour maintenir la production et la dépréciation du projet sur un nombre donné d'années.  

Le problème comporte 11 contraintes de type \textsf{K} et un nombre indéfini de type \textsf{H}. Les contraintes cachées sont des contraintes de type \textsf{NUSH} (Non-quatifiable, non-relaxable,, encourue durant la simulation et cachée). Les contraintes connues sont explicitées au tableau \ref{tab:sty} avec leur taxonomie respective.
\begin{table}[h]
\centering
\begin{tabular}{||r l l l||}
\hline
\# & Type & Description & Taxonomie\\
\hhline{||====||}
1 & Simulateur & Succès de la simulation & \textsf{NUSK}\\
2 & Procédé & Structure du séparateur à styrène acceptable & \textsf{NUSK}\\
3 & ~ & Structure du séparateur à benzène acceptable & \textsf{NUSK}\\
4 & ~ & Combustion des gaz possible et réglementaire & \textsf{NUSK}\\
5 & ~ & Borne minimale sur la pureté du styrène produit & \textsf{QRSK}\\
6 & ~ & Borne minimale sur la pureté du benzène produit & \textsf{QRSK}\\
7 & ~ & Borne minimale sur la conversion des résidus en styrène & \textsf{QRSK}\\	
8 & Économique & Borne maximale sur le temps requis pour atteindre la rentabilité & \textsf{QRSK}\\	
9 & ~ & Borne minimale sur les flux de trésorerie actualisés & \textsf{QRSK}\\
10 & ~ & Borne maximale sur l'investissement total & \textsf{QRSK}\\
11 & ~ & Borne maximale sur le coût annuel & \textsf{QRSK}\\
\hline
\end{tabular}
\caption{Contraintes de STYRENE groupées par type}\label{tab:sty}
\end{table}
\subsection{LOCKWOOD}\label{sec:loc}
description + pre analyse des stratégies
\section{Méthodologie}\label{sec:met}
Cette section décrit la méthodologie employée pour obtenir les algorithmes et leurs implémentations opportunistes.
\subsection{NOMAD}\label{sec:nom}
L'implémentation des l'algorithme \CS, \GPS et \MADS utilisée est NOMAD 3.8.1~\cite{Le09b} (Non-linear Optimization with the Mesh Adaptive Direct search), disponible sur son site officiel~\cite{AuCo04a}. Initialement une implémentation de \MADS pour la recherche non-linéaire, NOMAD est aujourd'hui un des solveurs principaux dans la littérature pour l'optimisation de boîtes noires~\cite{CoScVibook}, ~\cite{RiSa2010} à l'aide de recherche directe. L'architecture de NOMAD est telle que chaque point $t \in P^k$ ou $t \in S^k$ doit figurer sur le maillage $M^k:=\{x^k+\delta^kDy:y\in \N^p\}$ et ce même pour les algorithmes dont l'analyse de convergence n'est pas basée sur les contractions successives d'un maillage, soient \CS et \GPS. Chaque point déterminé avec sa direction correspondante $t = x^k + \delta^k d : d \in D^k$ qui ne corresponds pas à un point sur le maillage sera remplacé par un point sur le maillage $\tilde{t} \in M^k$ correspondant au point existant sur le maillage le plus près. Un paramétrage spécifique est nécessaire à NOMAD afin que le logiciel soit en mesure d'appliquer \CS, \GPS et \MADS selon leurs définitions des sections \ref{sec:cs}, \ref{sec:gps} et \ref{sec:mad}. L'ensemble des descriptions des différents paramètres sont issues de~\cite{Le09a}.   
  
Afin de ne pas obtenir le comportement d'un maillage granulaire, la valeur \texttt{XMESH}, qui corresponds au maillage anisotrope décrit dans~\cite{AuLedTr2014} est affectée au paramètre \texttt{MESH\_TYPE}.
\subsubsection{\CS}\label{sec:ncs}
La valeur \texttt{GPS 2N STATIC} est donnée au paramètre \texttt{DIRECTION\_TYPE}, qui contrôle l'algorithme et la mécanique gérant la génération des directions. En ayant \texttt{GPS} comme méthode de génération de directions, on a qu'un seul paramètre pour les directions, soit $\delta^k$, la longueur du pas. En ayant $2n$ directions définies de manière statique (en opposition à aléatoire), le logiciel génère toujours les mêmes directions, qui sont par défaut les directions coordonnées. Pour l'option \texttt{GPS} dans NOMAD, le maillage est obligatoirement isotrope, ce qui corresponds à la définition de \CS et qui annule l'anisotropie induite par l'option \texttt{XMESH}. 
 
Puisque l'ajustement du maillage dans \CS ne se fait qu'en cas de sonde fructueuse, on affecte au paramètre \texttt{MESH\_COARSENING\_COEFFICIENT} la valeur \texttt{0}.  
  
L'implémentation de l'opportunisme dans la recherche par coordonnées permets d'observer l'impact de la stratégie sur  la plus archaïque et la plus grossière des méthodes de recherche directe. Afin de désactiver les stratégies de recherche globale présente par défaut dans NOMAD, les certains paramètres se doivent d'être désactivés.
\begin{table}[h]
	\centering
	\begin{tabular}{||l l l||}
		\hline
		Nom & Description & Valeur\\
		\hhline{||===||}
		\texttt{MODEL\_SEARCH} & \SEARCH basée sur des modèles quadratiques~\cite{CoLed2011}& \texttt{no}\\
		\texttt{SPECULATIVE\_SEARCH} & \emph{Dynamic search} de \cite{AuDe2006} & \texttt{no}\\
		\texttt{SNAP\_TO\_BOUND} & Accolement des points aux bornes & \texttt{no}\\
		\hline
	\end{tabular}
	\caption{Paramètres de \SEARCH désactivés pour \CS dans NOMAD}\label{tab:pdc}
\end{table}
\subsubsection{GPS}\label{sec:ngp}
L'algorithme de \GPS a été obtenu a l'aide d'un paramétrage similaire à celui de \CS. Afin de correspondre à la définition de l'auteure~\cite{Torc97a} sans imposer de variations arbitraires, on utilise les directions coordonnées en affectant \texttt{GPS 2N STATIC} à \texttt{DIRECTION\_TYPE}, qui annule encore une fois l'anisotropie du maillage.

La différence se situe au niveau du \texttt{MESH\_REFINING\_COEFFICIENT} prenant la valeur \texttt{1}, qui corresponds à la puissance affectant le paramètre d'ajustement de maillage en cas de succès $\tau^{-1}$ de l'Algorithme \ref{alg:gps}. En laissant la valeur par défaut, on a que l'exposant $-1$ affecte le $\tau = 2$, contrairement à l'exposant $0$ dans le cas de \CS.

Étant donné la volonté, en premier lieu, d'isoler les étapes de \POLL des algorithmes, les mêmes paramètres figurant dans le tableau \ref{tab:pdc} sont désactivés pour \GPS.
\subsubsection{MADS}\label{sec:nma}
L'implémetation de \MADS est \textsf{OrthoMADS}~\cite{AbAuDeLe09} avec $n+1$ directions obtenue avec l'affection de \texttt{ORTHO N+1 NEG}. Il s'agit d'une instance de \MADS déterministe produisant $n$ direction orthogonales. La génération des directions $\D_\Delta^k$ est effectuée en quatre étapes.
\begin{enumerate}[label=\roman*.]
\item La séquence quasi-aléatoire de Halton~\cite{Ha60} produit un vecteur $u_{t,p} \in [0,1]^n$, avec $t$ l'indice de Halton.
\item Le vecteur $u_{t}$ est ensuite arrondie et accolée au maillage, pour être transformé en une direction de Halton ajustée $q_{t,l}$, avec $t$ l'indice de Halton du terme et $l$ l'indice du maillage.
\item La direction $q_{t,l}$ subit une transformation de Householder~\cite{Ho58}. Cette transformation génère une base orthogonale positive de $\R^n$ composée de vecteurs d'entiers, qu'on dénote $[H_{t,l}]$.
\end{enumerate}
On choisit le nombre $n+1$ de directions afin d'avoir le minimum de directions dans l'ensemble générateur positif, ce qui accélère la convergence. On choisit \texttt{NEG} pour indiquer au logiciel de fabriquer la $(n+1)$-ième direction en prenant la direction opposée à la somme des directions issues de la transformation de Householder $d_{n+1}=-\sum_{d\in H_{t,l}}$. L'ensemble $D^k_\Delta = [H_{t,l}~d_{n+1}]$ est ainsi une base positive et donc un ensemble générateur positif, condition nécessaire pour \MADS tel qu'évoqué à la section \ref{sec:mad}. On choisit cette méthode à celle de la complétion par modèle quadratique proposée par Audet et al.~\cite{AuIaLeDTr2014} par soucis d'interférence avec les paramètres nécessaires à l'obtention des stratégies d'ordonnancement. 

Les valeurs des paramètres du maillages et celles des paramètres figurant dans le tableau \ref{tab:pdc} sont identiques à celles de \GPS.
\subsubsection{MADS par défaut de NOMAD}\label{sec:ntr}
Afin d'observer l'impact de l'ordonnancement sur la résolution complète des problèmes d'optimisation identifiés, on résout l'ensemble des problèmes décrits à la section \ref{sec:pro} avec une instance de \MADS en communicant à NOMAD seulement les valeurs de paramètre indispensables. Ceci implique l'activation des étapes de \SEARCH désactivées pour les autres algorithmes, soient ceux du Tableau \ref{tab:pdc}, l'utilisation du maillage granulaire \texttt{GMESH}~\cite{Le09a} et la génération de direction \texttt{DIRECTION\_TYPE ORTHO N+1 QUAD}. Cette méthode de génération de directions indique la complétion de la base positive issue de Householder avec une direction $d_{n+1}$ qui permet l'obtention d'un ensemble générateur positif tout en minimisant l'approximation de $f$ au point candidat $x^k+\delta^k d_{n+1}$.   
Les résultats obtenus pour cet algorithme ne sont pas à l'abris d'interférences entre les paramètres nécessaires pour l'obtention des différentes stratégies d'ordonnancement et les paramètres de génération de directions. Ils sont aussi grandement influencés par les heuristiques présentes dans les étapes de \SEARCH. Dans ces mesures, on utilisera ces résultats avec de visualiser l'apport possible de l'opportunisme et de l'ordonnancement sur une résolution réaliste, plutôt que sur l'étape de sonde seulement.
\subsubsection{Opportunisme et stratégies d'ordonnancement}\label{sec:mop}
NOMAD permet à l'utilisateur de déterminer si la \POLL ou la \SEARCH se doit d'être opportuniste. Le paramètre nécessaire à la sonde opportuniste est \texttt{OPPORTUNISTIC\_EVAL}, pour lequel la valeur \texttt{YES} implique l'opportunisme simple tel que décrit dans la Définition~\ref{def:ops} et la valeur \texttt{NO} implique la sonde complète de la Définition~\ref{def:completepoll}. L'obtention des autres déclinaisons de la stratégie opportuniste se fait à l'aide des paramètres \texttt{OPPORTUNISTIC\_MIN\_NB\_SUCCES} suivi de son argument pour la stratégie opportuniste au $p^{\text{ème}}$ succès de la Définition~\ref{def:oppemesucces} et du paramètre \texttt{OPPORTUNISTIC\_MIN\_EVAL} suivi de son paramètre pour l'obtention de la stratégie opportuniste avec un minimum de $q$ évaluations de la Définition \ref{def:opmineval}.

Les stratégies d'ordonnancement lexicographique, en fonction de la direction du dernier succès et en fonction d'un modèle sont implémentées dans NOMAD et accessibles à l'utilisateur avec la bonne combinaison de paramètres. Le Tableau~\ref{tab:ppo} décrit les combinaisons nécessaires à l'obtention des stratégies.
\begin{table}[h]
	\centering
	\begin{tabular}{||l|c|c|c||}
		\hline
		Paramètre & \texttt{OPPORTUNISTIC\_EVAL} & \texttt{DISABLE} & \texttt{MODEL\_EVAL\_SORT} \\
		\hhline{||====||}
		Sonde complète & \texttt{NO} & N/A & N/A\\
		Lexicographique & \texttt{YES} & \texttt{EVAL\_SORT} & \texttt{NO}\\
		Dernier succès & \texttt{YES} & N/A & \texttt{NO}\\
		Avec modèles & \texttt{YES} & N/A & \texttt{YES}\\
		\hline
	\end{tabular}
	\caption{Paramètres pour l'obtention de certaines stratégies d'ordonnancement}\label{tab:ppo}
\end{table}
Les stratégies restantes sont obtenues à l'aide de l'activation du paramètre \texttt{SGTE\_EVAL\_SORT}. La stratégie omnisciente est obtenue en attribuant au paramètre \texttt{SGTE\_EXE} l'exécutable destiné à servir même de la boîte noire. Dans le même ordre d'esprit, la stratégie négative-omnisciente est obtenue en fournissant à ce même paramètre le nom d'un exécutable retournant $\tilde{f}:=-f(x)$ en argument. Enfin, on lui donne un exécutable retournant un nombre aléatoire en argument pour obtenir la stratégie d'ordonnancement aléatoire.
\subsection{GSS}\label{sec:hop}
La comparaison de certaines stratégies dans \GSS est possible grace à son implémentation présente dans le logiciel d'optimisation HOPSPACK 2.0~\cite{Plan09,GrKoLe2008}, héritier de APPSPACK~\cite{GrKo06}. Le logiciel est avant tout destiné à l'interfaçage de différents algorithmes d'optimisation de tout genre en parallèle, mais contient une implémentation de \GSS présente par défaut. La gestion des contraintes est possible seulement si les contraintes sont formulables dans leurs formes analytiques, ce qui ne convient pas à la définition du domaine d'optimisation sans-dérivées étudié dans cet ouvrage.

L'implémention de \GSS de HOPSPACK est opportuniste simple, sans option pour désactiver la stratégie. Cependant, l'implémentation offre de changer la changer la stratégie d'ordonnancement à l'aide du paramètre \texttt{Use Random Order}, pour lequel l'argument \texttt{False} entraine un ordonnancement déterministe non précisée, tandis que \texttt{True} entraine la stratégie d'ordonnancement aléatoire.
\subsection{Implicit Filtering}\label{sec:mim}
L'algorithme \imfil est indissociable de son implémentation en MATLAB, dénommée imfil 1.0~\cite{Kelley2011,Kelley2011b}. Il s'agit d'une implémentation de l'algorithme définit à la section \ref{sec:imf} munie d'un méchanisme de gestion d'échec de la fonction objectif à retourner une valeur, ce qui en fait un solveur applicable aux boîtes noires. Ces contraintes cachées sont traitées avec un procédé similaire à la barrière extrême. L'implémentation nécessite que des bornes soit fournies pour chaque problème.

Tel que mentionné à la Section~\ref{sec:mis}, l'auteur et architecte de l'implémentation C.T. Kelley reconnaît la pertinence de l'opportunisme mais ne l'incorpore pas dans son implémentation. L'implémentation de la stratégie est effectuée en modifiant les portions \texttt{imfil\_poll\_stencil} pour y implémenter les stratégies opportunistes et d'ordonnancement voulues. L'idée principale de l'implémentation de l'opportunisme dans \imfil est qu'on y simule que les points non-évalué suivant l'arrêt prématuré de la sonde ont échoués à retourner une valeur. L'implémentation procède ensuite au \textsf{BLS} avec seulement une portion des valeurs aux points évalués. Les points artificiellement communiqués comme des échecs sont réinitialisés avant la prochaine itération.
\section{Comparaison des stratégies algorithmiques}\label{sec:com}
Le banc d'essai comporte une banque de problèmes analytique non-contraints, une banque de problèmes contraints et des exemples de boîtes noires à haut coût d'évaluation. 

Les résolutions sont effectuées avec un budget d'évaluations de $1000(n+1)$. Les profils de donnés présents dans le corps du travail sont tracés avec les seuils de tolérance $\tau = 10^{-3}$ et $\tau = 10^{-7}$ pour avoir un aperçu des résolutions à basse et haute précision. Chaque combinaison de problème, de stratégie opportuniste et de stratégie d'ordonnancement sont résolues avec 10 germes aléatoires différentes. Les algorithmes issus de NOMAD qui peuvent bénéficier de la gestion des contraintes par barrière progressive sont testés sur chaque banque de problème. Pour la stratégie opportuniste de la Définition \ref{def:oppemesucces}, on statue que $p=2$, afin d'observer l'arrêt de la sonde après deux succès. Pour la stratégie opportuniste de la Définition \ref{def:opmineval}, on déterminera que $q = \text{ceil}(n/2)$, afin de permettre à un algorithme d'explorer au moins un orthant.

En raison de l'incapacité de leurs implémentations à gérer les contraintes de type boîte noire, \GSS et \imfil sont testées seulement avec l'ensemble de problèmes Moré-Wild. Les profils de données incluent les 212 instances possibles des problèmes Moré-Wild. Les détails des résolutions par type de problème et à différentes précisions sont fournis en annexe.
\subsection{CS}\label{sec:ccs}
	\begin{figure}[!htb]\label{fig:cs_mw}
		\centering
		\begin{subfigure}{0.43\textwidth}
			\includegraphics[width=\linewidth]{images/data_c_TOUS_1E-3_log.png}
			\label{fig:data_c_TOUS_1E-3_log}
		\end{subfigure}%\hspace*{\fill}
		\begin{subfigure}{0.43\textwidth}
			\includegraphics[width=\linewidth]{images/data_c_TOUS_1E-7_log.png}
			\label{fig:data_c_TOUS_1E-7_log}
		\end{subfigure}
		\smallskip
		\begin{subfigure}{0.95\textwidth}
			\includegraphics[width=\linewidth]{images/legende_nomad.png}
		\end{subfigure}
		\caption{Comparaison sur problèmes lisses de Moré-Wild avec \CS}
	\end{figure}
La recherche par coordonnée offre une opportunité sans pareille d'observer l'impact de l'ordonnancement. Puisqu'il s'agit essentiellement d'une étape de sonde rudimentaire, elle sera très sensible à l'opportunisme et à la stratégie d'ordonnancement qui le guidera. On s'attend ici à des profils très distincts. Sur les problèmes lisses, tel qu'observable à l'Annexe~\ref{ann:A}, les algorithmes se comportent en moyenne de façon plus stable.

Outre de rappeler la pertinence des profils de données, les profils de performances montrent que la stratégie aléatoire et la stratégie d'ordonnancement par modèles sont de performances comparables. On peut y comprendre ici que les stratégies sont uniquement comparées à la stratégie omnisciente pour la tolérance la plus haute $\tau = 0.1$. La marge d'erreur est aussi limitée pour les autres valeurs de $\tau = 0.01$ et $\tau = 0.001$, alors que pour près de $95\%$ on utilise la stratégie omnisciente pour comparer le résultat. Dans cette situation, on peut prendre directement l'allure du profil pour hiérarchiser les stratégies sur l'ensemble de problème sans se soucier de l'erreur induite par la nature des profils de performance~\cite{GoSc2016}. Sur les profils de performance avec $\tau = 0.1$, on voit que la stratégie d'ordonnancement en fonction du dernier succès est supérieure à la stratégie aléatoire jusqu'à un ratio de performance avoisinant $2$, après coup elle est reléguée à la quatrième plus performante. Malgré que cette stratégie soit plus raffinée que la stratégie aléatoire, elle ne prendra avantage de la forme du problème que si celui-ci n'est pas composé de plusieurs minimums locaux et de points de selle. On pourrait conclure que l'ensemble de problèmes utilisé ici possède une proportion de problèmes dont la structure n'est pas idéale pour un ordonnancement basé uniquement sur le succès précédent. La stratégie lexicographique n'est pas comparable en terme de performance. La stratégie pousse l'algorithme à épuiser ses sources de directions de déscente une par une. Au fil des évaluations, les directions de déscente déjà épuisées seront évaluées en début de sonde, et celles prometteuses seront toujours à la fin de la liste, ce qui aura pour effet de décaler les succès de plus en plus au cours du déroulement de l'algorithme. Pour sa part, la sonde complète ne peut pas compétitionner en observant les profils de performance, puisqu'elle prends un nombre $n$ fois le nombre d'évaluation nécéssité par la stratégie omnisciente. La stratégie négative omnisciente n'est pas appropriée dans la comparaison avec des profils de performance. Ces tendances sont observables aussi pour les ratios de tolérance moins élevés, qui figurent dans le deuxième et le troisième tracé des profils de performance. On en conclu que les solutions trouvées avec toutes les méthodes sont soit exacte à $0.01\%$ ou alors à plus de $10\%$ de différence de celle déterminée avec la stratégie omnisciente.

Dans le cas des tests avec cette famille d'algorithme, l'utilisation de la stratégie omnisciente requiert le traçage de profils de données pour bien pouvoir comparer les stratégies.  Ce type de métrique est choisi plutôt que les profils de performance car la nature de leur abscisse est relative au nombre d'itérations avant d'atteindre le minimum, permettant ainsi de juger de la performance d'un algorithme indépendemment de ses concurrents. Ainsi, les profils de données permettent de mieux distinguer le meilleur performant sans qu'il soit joint aux axes, tel qu'illustré dans les trois profils de performances de la figure précédente. Nous savons d'ailleurs des profils de performance que la stratégie omnisciente donne toujours le meilleur rendement. Ainsi, on peut se fier directement à l'apparence des courbes.

Premièrement, la stratégie omnisciente semble toujours la meilleure, alors qu'on peut toutefois déterminer avec quelle ampleur elle domine les autres. On explique que son ordonnée n'atteint pas $1.0$ par le fait que seulement jusqu'à $70 \times n+1$ évaluations sont illustrées. Pour des problèmes ayant jusqu'à $n=12$ variables tels qu'il en existe dans la collection de Moré-Wild utilisée, on a que $70$ gradients simplex correspondent à $910$ évaluations de la boîte noire, qui peuvent ne pas être suffisantes pour l'obtention de la meilleure solution obtenue. On observe que les comparaisons des stratégies réalistes faitent avec les profils de performance ne sont pas remis en question avec les profils de données outre la stratégie omnisciente.
	\subsection{GPS}\label{sec:cgp}
%	\begin{figure}[!htb] % GPS-SMOOTH
%		\centering
%		\begin{subfigure}{0.43\textwidth}
%			\includegraphics[width=\linewidth]{images/perf_g_SMOOTH_01_log.png}
%			%\caption{First subfigure} \label{fig:a}
%		\end{subfigure}%\hspace*{\fill}
%		\begin{subfigure}{0.43\textwidth}
%			\includegraphics[width=\linewidth]{images/data_g_SMOOTH_01_log.png}
%			%\caption{Second subfigure} \label{fig:b}
%		\end{subfigure}
%		\smallskip
%		\begin{subfigure}{0.43\textwidth}
%			\includegraphics[width=\linewidth]{images/perf_g_SMOOTH_001_log.png}
%			%\caption{Third subfigure} \label{fig:c}
%		\end{subfigure}%\hspace*{\fill}
%		\begin{subfigure}{0.43\textwidth}
%			\includegraphics[width=\linewidth]{images/data_g_SMOOTH_001_log.png}
%			%\caption{Fourth subfigure} \label{fig:d}
%		\end{subfigure}
%		\smallskip
%		\begin{subfigure}{0.43\textwidth}
%			\includegraphics[width=\linewidth]{images/perf_g_SMOOTH_0001_log.png}
%			%\caption{Third subfigure} \label{fig:e}
%		\end{subfigure}%\hspace*{\fill}
%		\begin{subfigure}{0.43\textwidth}
%			\includegraphics[width=\linewidth]{images/data_g_SMOOTH_0001_log.png}
%			%\caption{Fourth subfigure} \label{fig:f}
%		\end{subfigure}
%		\smallskip
%		\begin{subfigure}{0.95\textwidth}
%			\includegraphics[width=\linewidth]{images/legende_nomad.png}
%		\end{subfigure}
%		\caption{Comparaison sur problèmes lisses de Moré-Wild avec \CS} \label{fig:1}
%	\end{figure}
%	\clearpage
%	\begin{figure}[!htb] % CS-SMOOTH
%		\centering
%		\begin{subfigure}{0.43\textwidth}
%			\includegraphics[width=\linewidth]{images/perf_g_NONDIFF_01_log.png}
%			%\caption{First subfigure} \label{fig:a}
%		\end{subfigure}%\hspace*{\fill}
%		\begin{subfigure}{0.43\textwidth}
%			\includegraphics[width=\linewidth]{images/data_g_NONDIFF_01_log.png}
%			%\caption{Second subfigure} \label{fig:b}
%		\end{subfigure}
%		\smallskip
%		\begin{subfigure}{0.43\textwidth}
%			\includegraphics[width=\linewidth]{images/perf_g_NONDIFF_001_log.png}
%			%\caption{Third subfigure} \label{fig:c}
%		\end{subfigure}%\hspace*{\fill}
%		\begin{subfigure}{0.43\textwidth}
%			\includegraphics[width=\linewidth]{images/data_g_NONDIFF_001_log.png}
%			%\caption{Fourth subfigure} \label{fig:d}
%		\end{subfigure}
%		\smallskip
%		\begin{subfigure}{0.43\textwidth}
%			\includegraphics[width=\linewidth]{images/perf_g_NONDIFF_0001_log.png}
%			%\caption{Third subfigure} \label{fig:e}
%		\end{subfigure}%\hspace*{\fill}
%		\begin{subfigure}{0.43\textwidth}
%			\includegraphics[width=\linewidth]{images/data_g_NONDIFF_0001_log.png}
%			%\caption{Fourth subfigure} \label{fig:f}
%		\end{subfigure}
%		\smallskip
%		\begin{subfigure}{0.95\textwidth}
%			\includegraphics[width=\linewidth]{images/legende_nomad.png}
%		\end{subfigure}
%		\caption{Comparaison sur problèmes non-differentiables de Moré-Wild avec \CS} \label{fig:1}
%	\end{figure}
%	\clearpage
%	\begin{figure}[!htb] % CS-SMOOTH
%		\centering
%		\begin{subfigure}{0.43\textwidth}
%			\includegraphics[width=\linewidth]{images/perf_g_NOISY3_01_log.png}
%			%\caption{First subfigure} \label{fig:a}
%		\end{subfigure}%\hspace*{\fill}
%		\begin{subfigure}{0.43\textwidth}
%			\includegraphics[width=\linewidth]{images/data_g_NOISY3_01_log.png}
%			%\caption{Second subfigure} \label{fig:b}
%		\end{subfigure}
%		\smallskip
%		\begin{subfigure}{0.43\textwidth}
%			\includegraphics[width=\linewidth]{images/perf_g_NOISY3_001_log.png}
%			%\caption{Third subfigure} \label{fig:c}
%		\end{subfigure}%\hspace*{\fill}
%		\begin{subfigure}{0.43\textwidth}
%			\includegraphics[width=\linewidth]{images/data_g_NOISY3_001_log.png}
%			%\caption{Fourth subfigure} \label{fig:d}
%		\end{subfigure}
%		\smallskip
%		\begin{subfigure}{0.43\textwidth}
%			\includegraphics[width=\linewidth]{images/perf_g_NOISY3_0001_log.png}
%			%\caption{Third subfigure} \label{fig:e}
%		\end{subfigure}%\hspace*{\fill}
%		\begin{subfigure}{0.43\textwidth}
%			\includegraphics[width=\linewidth]{images/data_g_NOISY3_0001_log.png}
%			%\caption{Fourth subfigure} \label{fig:f}
%		\end{subfigure}
%		\smallskip
%		\begin{subfigure}{0.95\textwidth}
%			\includegraphics[width=\linewidth]{images/legende_nomad.png}
%		\end{subfigure}
%		\caption{Comparaison sur problèmes bruités de Moré-Wild avec \CS} \label{fig:3}
%	\end{figure}
%	\clearpage
	Commentaires Commentaires Commentaires Commentaires Commentaires Commentaires Commentaires Commentaires Commentaires Commentaires Commentaires Commentaires Commentaires Commentaires Commentaires Commentaires Commentaires Commentaires Commentaires Commentaires Commentaires Commentaires Commentaires Commentaires Commentaires Commentaires Commentaires Commentaires Commentaires Commentaires Commentaires Commentaires Commentaires Commentaires Commentaires Commentaires Commentaires Commentaires Commentaires Commentaires Commentaires Commentaires 
	\clearpage
\subsection{MADS}\label{sec:cma}
%		\subsubsection{Moré-Wild}
%			\begin{figure}[!htb] % CS-SMOOTH
%				\centering
%				\begin{subfigure}{0.43\textwidth}
%					\includegraphics[width=\linewidth]{images/perf_m_SMOOTH_01_log.png}
%					%\caption{First subfigure} \label{fig:a}
%				\end{subfigure}%\hspace*{\fill}
%				\begin{subfigure}{0.43\textwidth}
%					\includegraphics[width=\linewidth]{images/data_m_SMOOTH_01_log.png}
%					%\caption{Second subfigure} \label{fig:b}
%				\end{subfigure}
%				\smallskip
%				\begin{subfigure}{0.43\textwidth}
%					\includegraphics[width=\linewidth]{images/perf_m_SMOOTH_001_log.png}
%					%\caption{Third subfigure} \label{fig:c}
%				\end{subfigure}%\hspace*{\fill}
%				\begin{subfigure}{0.43\textwidth}
%					\includegraphics[width=\linewidth]{images/data_m_SMOOTH_001_log.png}
%					%\caption{Fourth subfigure} \label{fig:d}
%				\end{subfigure}
%				\smallskip
%				\begin{subfigure}{0.43\textwidth}
%					\includegraphics[width=\linewidth]{images/perf_m_SMOOTH_0001_log.png}
%					%\caption{Third subfigure} \label{fig:e}
%				\end{subfigure}%\hspace*{\fill}
%				\begin{subfigure}{0.43\textwidth}
%					\includegraphics[width=\linewidth]{images/data_m_SMOOTH_0001_log.png}
%					%\caption{Fourth subfigure} \label{fig:f}
%				\end{subfigure}
%				\smallskip
%				\begin{subfigure}{0.95\textwidth}
%					\includegraphics[width=\linewidth]{images/legende_nomad.png}
%				\end{subfigure}
%				\caption{Comparaison sur problèmes lisses de Moré-Wild avec \MADS} \label{fig:1}
%			\end{figure}
%			\clearpage
%			\begin{figure}[!htb] % CS-SMOOTH
%				\centering
%				\begin{subfigure}{0.43\textwidth}
%					\includegraphics[width=\linewidth]{images/perf_m_NONDIFF_01_log.png}
%					%\caption{First subfigure} \label{fig:a}
%				\end{subfigure}%\hspace*{\fill}
%				\begin{subfigure}{0.43\textwidth}
%					\includegraphics[width=\linewidth]{images/data_m_NONDIFF_01_log.png}
%					%\caption{Second subfigure} \label{fig:b}
%				\end{subfigure}
%				\smallskip
%				\begin{subfigure}{0.43\textwidth}
%					\includegraphics[width=\linewidth]{images/perf_m_NONDIFF_001_log.png}
%					%\caption{Third subfigure} \label{fig:c}
%				\end{subfigure}%\hspace*{\fill}
%				\begin{subfigure}{0.43\textwidth}
%					\includegraphics[width=\linewidth]{images/data_m_NONDIFF_001_log.png}
%					%\caption{Fourth subfigure} \label{fig:d}
%				\end{subfigure}
%				\smallskip
%				\begin{subfigure}{0.43\textwidth}
%					\includegraphics[width=\linewidth]{images/perf_m_NONDIFF_0001_log.png}
%					%\caption{Third subfigure} \label{fig:e}
%				\end{subfigure}%\hspace*{\fill}
%				\begin{subfigure}{0.43\textwidth}
%					\includegraphics[width=\linewidth]{images/data_m_NONDIFF_0001_log.png}
%					%\caption{Fourth subfigure} \label{fig:f}
%				\end{subfigure}
%				\smallskip
%				\begin{subfigure}{0.95\textwidth}
%					\includegraphics[width=\linewidth]{images/legende_nomad.png}
%				\end{subfigure}
%				\caption{Comparaison sur problèmes non-differentiables de Moré-Wild avec \CS} \label{fig:1}
%			\end{figure}
%			\clearpage
%			\begin{figure}[!htb] % CS-SMOOTH
%				\centering
%				\begin{subfigure}{0.43\textwidth}
%					\includegraphics[width=\linewidth]{images/perf_m_NOISY3_01_log.png}
%					%\caption{First subfigure} \label{fig:a}
%				\end{subfigure}%\hspace*{\fill}
%				\begin{subfigure}{0.43\textwidth}
%					\includegraphics[width=\linewidth]{images/data_m_NOISY3_01_log.png}
%					%\caption{Second subfigure} \label{fig:b}
%				\end{subfigure}
%				\smallskip
%				\begin{subfigure}{0.43\textwidth}
%					\includegraphics[width=\linewidth]{images/perf_m_NOISY3_001_log.png}
%					%\caption{Third subfigure} \label{fig:c}
%				\end{subfigure}%\hspace*{\fill}
%				\begin{subfigure}{0.43\textwidth}
%					\includegraphics[width=\linewidth]{images/data_m_NOISY3_001_log.png}
%					%\caption{Fourth subfigure} \label{fig:d}
%				\end{subfigure}
%				\smallskip
%				\begin{subfigure}{0.43\textwidth}
%					\includegraphics[width=\linewidth]{images/perf_m_NOISY3_0001_log.png}
%					%\caption{Third subfigure} \label{fig:e}
%				\end{subfigure}%\hspace*{\fill}
%				\begin{subfigure}{0.43\textwidth}
%					\includegraphics[width=\linewidth]{images/data_m_NOISY3_0001_log.png}
%					%\caption{Fourth subfigure} \label{fig:f}
%				\end{subfigure}
%				\smallskip
%				\begin{subfigure}{0.95\textwidth}
%					\includegraphics[width=\linewidth]{images/legende_nomad.png}
%				\end{subfigure}
%				\caption{Comparaison sur problèmes bruités de Moré-Wild avec \CS} \label{fig:3}
%			\end{figure}
%			\clearpage
%			Commentaires Commentaires Commentaires Commentaires Commentaires Commentaires Commentaires Commentaires Commentaires Commentaires Commentaires Commentaires Commentaires Commentaires Commentaires Commentaires Commentaires Commentaires Commentaires Commentaires Commentaires Commentaires Commentaires Commentaires Commentaires Commentaires Commentaires Commentaires Commentaires Commentaires Commentaires Commentaires Commentaires Commentaires Commentaires Commentaires Commentaires Commentaires Commentaires Commentaires Commentaires Commentaires 
%			\clearpage
%		\subsubsection{STYRENE}
%			\begin{figure}[!htb] % CS-SMOOTH
%				\centering
%				\begin{subfigure}{0.43\textwidth}
%					\includegraphics[width=\linewidth]{images/conv_m_STYRENE_n_oo.png}
%					%\caption{First subfigure} \label{fig:a}
%				\end{subfigure}%\hspace*{\fill}
%				\begin{subfigure}{0.43\textwidth}
%					\includegraphics[width=\linewidth]{images/conv_m_STYRENE_0n_oo.png}
%					%\caption{Second subfigure} \label{fig:b}
%				\end{subfigure}
%				\smallskip
%				\begin{subfigure}{0.43\textwidth}
%					\includegraphics[width=\linewidth]{images/conv_m_STYRENE_or_om.png}
%					%\caption{Third subfigure} \label{fig:c}
%				\end{subfigure}%\hspace*{\fill}
%				\begin{subfigure}{0.43\textwidth}
%					\includegraphics[width=\linewidth]{images/conv_m_STYRENE_om_oo.png}
%					%\caption{Fourth subfigure} \label{fig:d}
%				\end{subfigure}
%				\smallskip
%				\begin{subfigure}{0.43\textwidth}
%					\includegraphics[width=\linewidth]{images/conv_m_STYRENE_om_n.png}
%					%\caption{Third subfigure} \label{fig:e}
%				\end{subfigure}%\hspace*{\fill}
%				\begin{subfigure}{0.43\textwidth}
%					\includegraphics[width=\linewidth]{images/conv_m_STYRENE_om_n.png}
%					%\caption{Fourth subfigure} \label{fig:f}
%				\end{subfigure}
%				\smallskip
%				\begin{subfigure}{0.95\textwidth}
%					\includegraphics[width=\linewidth]{images/legende_nomad.png}
%				\end{subfigure}
%				\caption{Comparaison sur STYRENE avec \MADS} \label{fig:3}
%			\end{figure}
%			\clearpage
			Commentaires CommentairesCommentaire sCommentairesCommentaire sCommentairesCommentairesCommentai resCommentairesCommentairesCommentairesCommentairesCommentairesComm entairesCommentairesCommentairesC ommentairesCommentairesCommentairesCommentairesCommentairesCommentair esCommentairesCommentairesCommen tairesCommentairesCommentairesCommentairesCommentairesCommentairesComme ntairesCommentairesCommentaires CommentairesCommentairesCommentairesCommentairesCommentairesCommentairesC ommentairesCommentairesComment airesCommentairesCommentairesCommentairesCommentairesCommentairesCommentair esCommentaires
			\clearpage
\subsection{\MADS de NOMAD}\label{sec:ctr}
%	\subsubsection{Moré-Wild}
%		\begin{figure}[!htb] % CS-SMOOTH
%			\centering
%			\begin{subfigure}{0.43\textwidth}
%				\includegraphics[width=\linewidth]{images/perf_m_SMOOTH_01_log.png}
%				%\caption{First subfigure} \label{fig:a}
%			\end{subfigure}%\hspace*{\fill}
%			\begin{subfigure}{0.43\textwidth}
%				\includegraphics[width=\linewidth]{images/data_m_SMOOTH_01_log.png}
%				%\caption{Second subfigure} \label{fig:b}
%			\end{subfigure}
%			\smallskip
%			\begin{subfigure}{0.43\textwidth}
%				\includegraphics[width=\linewidth]{images/perf_m_SMOOTH_001_log.png}
%				%\caption{Third subfigure} \label{fig:c}
%			\end{subfigure}%\hspace*{\fill}
%			\begin{subfigure}{0.43\textwidth}
%				\includegraphics[width=\linewidth]{images/data_m_SMOOTH_001_log.png}
%				%\caption{Fourth subfigure} \label{fig:d}
%			\end{subfigure}
%			\smallskip
%			\begin{subfigure}{0.43\textwidth}
%				\includegraphics[width=\linewidth]{images/perf_m_SMOOTH_0001_log.png}
%				%\caption{Third subfigure} \label{fig:e}
%			\end{subfigure}%\hspace*{\fill}
%			\begin{subfigure}{0.43\textwidth}
%				\includegraphics[width=\linewidth]{images/data_m_SMOOTH_0001_log.png}
%				%\caption{Fourth subfigure} \label{fig:f}
%			\end{subfigure}
%			\smallskip
%			\begin{subfigure}{0.95\textwidth}
%				\includegraphics[width=\linewidth]{images/legende_nomad.png}
%			\end{subfigure}
%			\caption{Comparaison sur problèmes lisses de Moré-Wild avec \MADS par défaut de NOMAD} \label{fig:1}
%		\end{figure}
%		\clearpage
%		\begin{figure}[!htb] % CS-SMOOTH
%			\centering
%			\begin{subfigure}{0.43\textwidth}
%				\includegraphics[width=\linewidth]{images/perf_m_NONDIFF_01_log.png}
%				%\caption{First subfigure} \label{fig:a}
%			\end{subfigure}%\hspace*{\fill}
%			\begin{subfigure}{0.43\textwidth}
%				\includegraphics[width=\linewidth]{images/data_m_NONDIFF_01_log.png}
%				%\caption{Second subfigure} \label{fig:b}
%			\end{subfigure}
%			\smallskip
%			\begin{subfigure}{0.43\textwidth}
%				\includegraphics[width=\linewidth]{images/perf_m_NONDIFF_001_log.png}
%				%\caption{Third subfigure} \label{fig:c}
%			\end{subfigure}%\hspace*{\fill}
%			\begin{subfigure}{0.43\textwidth}
%				\includegraphics[width=\linewidth]{images/data_m_NONDIFF_001_log.png}
%				%\caption{Fourth subfigure} \label{fig:d}
%			\end{subfigure}
%			\smallskip
%			\begin{subfigure}{0.43\textwidth}
%				\includegraphics[width=\linewidth]{images/perf_m_NONDIFF_0001_log.png}
%				%\caption{Third subfigure} \label{fig:e}
%			\end{subfigure}%\hspace*{\fill}
%			\begin{subfigure}{0.43\textwidth}
%				\includegraphics[width=\linewidth]{images/data_m_NONDIFF_0001_log.png}
%				%\caption{Fourth subfigure} \label{fig:f}
%			\end{subfigure}
%			\smallskip
%			\begin{subfigure}{0.95\textwidth}
%				\includegraphics[width=\linewidth]{images/legende_nomad.png}
%			\end{subfigure}
%			\caption{Comparaison sur problèmes non-differentiables de Moré-Wild avec \MADS par défaut de NOMAD} \label{fig:1}
%		\end{figure}
%		\clearpage
%		\begin{figure}[!htb] % CS-SMOOTH
%			\centering
%			\begin{subfigure}{0.43\textwidth}
%				\includegraphics[width=\linewidth]{images/perf_m_NOISY3_01_log.png}
%				%\caption{First subfigure} \label{fig:a}
%			\end{subfigure}%\hspace*{\fill}
%			\begin{subfigure}{0.43\textwidth}
%				\includegraphics[width=\linewidth]{images/data_m_NOISY3_01_log.png}
%				%\caption{Second subfigure} \label{fig:b}
%			\end{subfigure}
%			\smallskip
%			\begin{subfigure}{0.43\textwidth}
%				\includegraphics[width=\linewidth]{images/perf_m_NOISY3_001_log.png}
%				%\caption{Third subfigure} \label{fig:c}
%			\end{subfigure}%\hspace*{\fill}
%			\begin{subfigure}{0.43\textwidth}
%				\includegraphics[width=\linewidth]{images/data_m_NOISY3_001_log.png}
%				%\caption{Fourth subfigure} \label{fig:d}
%			\end{subfigure}
%			\smallskip
%			\begin{subfigure}{0.43\textwidth}
%				\includegraphics[width=\linewidth]{images/perf_m_NOISY3_0001_log.png}
%				%\caption{Third subfigure} \label{fig:e}
%			\end{subfigure}%\hspace*{\fill}
%			\begin{subfigure}{0.43\textwidth}
%				\includegraphics[width=\linewidth]{images/data_m_NOISY3_0001_log.png}
%				%\caption{Fourth subfigure} \label{fig:f}
%			\end{subfigure}
%			\smallskip
%			\begin{subfigure}{0.95\textwidth}
%				\includegraphics[width=\linewidth]{images/legende_nomad.png}
%			\end{subfigure}
%			\caption{Comparaison sur problèmes bruités de Moré-Wild avec \MADS par défaut de NOMAD} \label{fig:3}
%		\end{figure}
%		\clearpage
%		Commentaires Commentaires Commentaires Commentaires Commentaires Commentaires Commentaires Commentaires Commentaires Commentaires Commentaires Commentaires Commentaires Commentaires Commentaires Commentaires Commentaires Commentaires Commentaires Commentaires Commentaires Commentaires Commentaires Commentaires Commentaires Commentaires Commentaires Commentaires Commentaires Commentaires Commentaires Commentaires Commentaires Commentaires Commentaires Commentaires Commentaires Commentaires Commentaires Commentaires Commentaires Commentaires 
%		\clearpage
%	\subsubsection{STYRENE}
%		\begin{figure}[!htb] % CS-SMOOTH
%			\centering
%			\begin{subfigure}{0.43\textwidth}
%				\includegraphics[width=\linewidth]{images/conv_t_STYRENE_n_oo.png}
%				%\caption{First subfigure} \label{fig:a}
%			\end{subfigure}%\hspace*{\fill}
%			\begin{subfigure}{0.43\textwidth}
%				\includegraphics[width=\linewidth]{images/conv_t_STYRENE_0n_oo.png}
%				%\caption{Second subfigure} \label{fig:b}
%			\end{subfigure}
%			\smallskip
%			\begin{subfigure}{0.43\textwidth}
%				\includegraphics[width=\linewidth]{images/conv_t_STYRENE_or_om.png}
%				%\caption{Third subfigure} \label{fig:c}
%			\end{subfigure}%\hspace*{\fill}
%			\begin{subfigure}{0.43\textwidth}
%				\includegraphics[width=\linewidth]{images/conv_t_STYRENE_om_oo.png}
%				%\caption{Fourth subfigure} \label{fig:d}
%			\end{subfigure}
%			\smallskip
%			\begin{subfigure}{0.43\textwidth}
%				\includegraphics[width=\linewidth]{images/conv_t_STYRENE_om_n.png}
%				%\caption{Third subfigure} \label{fig:e}
%			\end{subfigure}%\hspace*{\fill}
%			\begin{subfigure}{0.43\textwidth}
%				\includegraphics[width=\linewidth]{images/conv_t_STYRENE_om_n.png}
%				%\caption{Fourth subfigure} \label{fig:f}
%			\end{subfigure}
%			\smallskip
%			\begin{subfigure}{0.95\textwidth}
%				\includegraphics[width=\linewidth]{images/legende_nomad.png}
%			\end{subfigure}
%			\caption{Comparaison sur STYRENE avec \MADS par défaut de NOMAD} \label{fig:3}
%		\end{figure}
%	\clearpage
	\subsection{\GSS}\label{sec:cgs}
%	\begin{figure}[!htb] % CS-SMOOTH
%		\centering
%		\begin{subfigure}{0.43\textwidth}
%			\includegraphics[width=\linewidth]{images/perf_gss_SMOOTH_01_log.png}
%			%\caption{First subfigure} \label{fig:a}
%		\end{subfigure}%\hspace*{\fill}
%		\begin{subfigure}{0.43\textwidth}
%			\includegraphics[width=\linewidth]{images/data_gss_SMOOTH_01_log.png}
%			%\caption{Second subfigure} \label{fig:b}
%		\end{subfigure}
%		\smallskip
%		\begin{subfigure}{0.43\textwidth}
%			\includegraphics[width=\linewidth]{images/perf_gss_SMOOTH_001_log.png}
%			%\caption{Third subfigure} \label{fig:c}
%		\end{subfigure}%\hspace*{\fill}
%		\begin{subfigure}{0.43\textwidth}
%			\includegraphics[width=\linewidth]{images/data_gss_SMOOTH_001_log.png}
%			%\caption{Fourth subfigure} \label{fig:d}
%		\end{subfigure}
%		\smallskip
%		\begin{subfigure}{0.43\textwidth}
%			\includegraphics[width=\linewidth]{images/perf_gss_SMOOTH_0001_log.png}
%			%\caption{Third subfigure} \label{fig:e}
%		\end{subfigure}%\hspace*{\fill}
%		\begin{subfigure}{0.43\textwidth}
%			\includegraphics[width=\linewidth]{images/data_gss_SMOOTH_0001_log.png}
%			%\caption{Fourth subfigure} \label{fig:f}
%		\end{subfigure}
%		\smallskip
%		\begin{subfigure}{0.95\textwidth}
%			\includegraphics[width=\linewidth]{images/legende_nomad.png}
%		\end{subfigure}
%		\caption{Comparaison sur problèmes lisses de Moré-Wild avec \MADS} \label{fig:1}
%	\end{figure}
%	\clearpage
%	\begin{figure}[!htb] % CS-SMOOTH
%		\centering
%		\begin{subfigure}{0.43\textwidth}
%			\includegraphics[width=\linewidth]{images/perf_gss_NONDIFF_01_log.png}
%			%\caption{First subfigure} \label{fig:a}
%		\end{subfigure}%\hspace*{\fill}
%		\begin{subfigure}{0.43\textwidth}
%			\includegraphics[width=\linewidth]{images/data_gss_NONDIFF_01_log.png}
%			%\caption{Second subfigure} \label{fig:b}
%		\end{subfigure}
%		\smallskip
%		\begin{subfigure}{0.43\textwidth}
%			\includegraphics[width=\linewidth]{images/perf_gss_NONDIFF_001_log.png}
%			%\caption{Third subfigure} \label{fig:c}
%		\end{subfigure}%\hspace*{\fill}
%		\begin{subfigure}{0.43\textwidth}
%			\includegraphics[width=\linewidth]{images/data_gss_NONDIFF_001_log.png}
%			%\caption{Fourth subfigure} \label{fig:d}
%		\end{subfigure}
%		\smallskip
%		\begin{subfigure}{0.43\textwidth}
%			\includegraphics[width=\linewidth]{images/perf_gss_NONDIFF_0001_log.png}
%			%\caption{Third subfigure} \label{fig:e}
%		\end{subfigure}%\hspace*{\fill}
%		\begin{subfigure}{0.43\textwidth}
%			\includegraphics[width=\linewidth]{images/data_gss_NONDIFF_0001_log.png}
%			%\caption{Fourth subfigure} \label{fig:f}
%		\end{subfigure}
%		\smallskip
%		\begin{subfigure}{0.95\textwidth}
%			\includegraphics[width=\linewidth]{images/legende_nomad.png}
%		\end{subfigure}
%		\caption{Comparaison sur problèmes non-differentiables de Moré-Wild avec \MADS} \label{fig:1}
%	\end{figure}
%	\clearpage
%	\begin{figure}[!htb] % CS-SMOOTH
%		\centering
%		\begin{subfigure}{0.43\textwidth}
%			\includegraphics[width=\linewidth]{images/perf_gss_NOISY3_01_log.png}
%			%\caption{First subfigure} \label{fig:a}
%		\end{subfigure}%\hspace*{\fill}
%		\begin{subfigure}{0.43\textwidth}
%			\includegraphics[width=\linewidth]{images/data_gss_NOISY3_01_log.png}
%			%\caption{Second subfigure} \label{fig:b}
%		\end{subfigure}
%		\smallskip
%		\begin{subfigure}{0.43\textwidth}
%			\includegraphics[width=\linewidth]{images/perf_gss_NOISY3_001_log.png}
%			%\caption{Third subfigure} \label{fig:c}
%		\end{subfigure}%\hspace*{\fill}
%		\begin{subfigure}{0.43\textwidth}
%			\includegraphics[width=\linewidth]{images/data_gss_NOISY3_001_log.png}
%			%\caption{Fourth subfigure} \label{fig:d}
%		\end{subfigure}
%		\smallskip
%		\begin{subfigure}{0.43\textwidth}
%			\includegraphics[width=\linewidth]{images/perf_gss_NOISY3_0001_log.png}
%			%\caption{Third subfigure} \label{fig:e}
%		\end{subfigure}%\hspace*{\fill}
%		\begin{subfigure}{0.43\textwidth}
%			\includegraphics[width=\linewidth]{images/data_gss_NOISY3_0001_log.png}
%			%\caption{Fourth subfigure} \label{fig:f}
%		\end{subfigure}
%		\smallskip
%		\begin{subfigure}{0.95\textwidth}
%			\includegraphics[width=\linewidth]{images/legende_nomad.png}
%		\end{subfigure}
%		\caption{Comparaison sur problèmes bruités de Moré-Wild avec \MADS} \label{fig:3}
%	\end{figure}
	\clearpage
	\subsection{\imfil}\label{sec:cim}
%	\begin{figure}[!htb] % CS-SMOOTH
%		\centering
%		\begin{subfigure}{0.43\textwidth}
%			\includegraphics[width=\linewidth]{images/perf_i_SMOOTH_01_log.png}
%			%\caption{First subfigure} \label{fig:a}
%		\end{subfigure}%\hspace*{\fill}
%		\begin{subfigure}{0.43\textwidth}
%			\includegraphics[width=\linewidth]{images/data_i_SMOOTH_01_log.png}
%			%\caption{Second subfigure} \label{fig:b}
%		\end{subfigure}
%		\smallskip
%		\begin{subfigure}{0.43\textwidth}
%			\includegraphics[width=\linewidth]{images/perf_i_SMOOTH_001_log.png}
%			%\caption{Third subfigure} \label{fig:c}
%		\end{subfigure}%\hspace*{\fill}
%		\begin{subfigure}{0.43\textwidth}
%			\includegraphics[width=\linewidth]{images/data_i_SMOOTH_001_log.png}
%			%\caption{Fourth subfigure} \label{fig:d}
%		\end{subfigure}
%		\smallskip
%		\begin{subfigure}{0.43\textwidth}
%			\includegraphics[width=\linewidth]{images/perf_i_SMOOTH_0001_log.png}
%			%\caption{Third subfigure} \label{fig:e}
%		\end{subfigure}%\hspace*{\fill}
%		\begin{subfigure}{0.43\textwidth}
%			\includegraphics[width=\linewidth]{images/data_i_SMOOTH_0001_log.png}
%			%\caption{Fourth subfigure} \label{fig:f}
%		\end{subfigure}
%		\smallskip
%		\begin{subfigure}{0.95\textwidth}
%			\includegraphics[width=\linewidth]{images/legende_nomad.png}
%		\end{subfigure}
%		\caption{Comparaison sur problèmes lisses de Moré-Wild avec \CS} \label{fig:1}
%	\end{figure}
%	\clearpage
%	\begin{figure}[!htb] % CS-SMOOTH
%		\centering
%		\begin{subfigure}{0.43\textwidth}
%			\includegraphics[width=\linewidth]{images/perf_i_NONDIFF_01_log.png}
%			%\caption{First subfigure} \label{fig:a}
%		\end{subfigure}%\hspace*{\fill}
%		\begin{subfigure}{0.43\textwidth}
%			\includegraphics[width=\linewidth]{images/data_i_NONDIFF_01_log.png}
%			%\caption{Second subfigure} \label{fig:b}
%		\end{subfigure}
%		\smallskip
%		\begin{subfigure}{0.43\textwidth}
%			\includegraphics[width=\linewidth]{images/perf_i_NONDIFF_001_log.png}
%			%\caption{Third subfigure} \label{fig:c}
%		\end{subfigure}%\hspace*{\fill}
%		\begin{subfigure}{0.43\textwidth}
%			\includegraphics[width=\linewidth]{images/data_i_NONDIFF_001_log.png}
%			%\caption{Fourth subfigure} \label{fig:d}
%		\end{subfigure}
%		\smallskip
%		\begin{subfigure}{0.43\textwidth}
%			\includegraphics[width=\linewidth]{images/perf_i_NONDIFF_0001_log.png}
%			%\caption{Third subfigure} \label{fig:e}
%		\end{subfigure}%\hspace*{\fill}
%		\begin{subfigure}{0.43\textwidth}
%			\includegraphics[width=\linewidth]{images/data_i_NONDIFF_0001_log.png}
%			%\caption{Fourth subfigure} \label{fig:f}
%		\end{subfigure}
%		\smallskip
%		\begin{subfigure}{0.95\textwidth}
%			\includegraphics[width=\linewidth]{images/legende_nomad.png}
%		\end{subfigure}
%		\caption{Comparaison sur problèmes non-differentiables de Moré-Wild avec \CS} \label{fig:1}
%	\end{figure}
%	\clearpage
%	\begin{figure}[!htb] % CS-SMOOTH
%		\centering
%		\begin{subfigure}{0.43\textwidth}
%			\includegraphics[width=\linewidth]{images/perf_i_NOISY3_01_log.png}
%			%\caption{First subfigure} \label{fig:a}
%		\end{subfigure}%\hspace*{\fill}
%		\begin{subfigure}{0.43\textwidth}
%			\includegraphics[width=\linewidth]{images/data_i_NOISY3_01_log.png}
%			%\caption{Second subfigure} \label{fig:b}
%		\end{subfigure}
%		\smallskip
%		\begin{subfigure}{0.43\textwidth}
%			\includegraphics[width=\linewidth]{images/perf_i_NOISY3_001_log.png}
%			%\caption{Third subfigure} \label{fig:c}
%		\end{subfigure}%\hspace*{\fill}
%		\begin{subfigure}{0.43\textwidth}
%			\includegraphics[width=\linewidth]{images/data_i_NOISY3_001_log.png}
%			%\caption{Fourth subfigure} \label{fig:d}
%		\end{subfigure}
%		\smallskip
%		\begin{subfigure}{0.43\textwidth}
%			\includegraphics[width=\linewidth]{images/perf_i_NOISY3_0001_log.png}
%			%\caption{Third subfigure} \label{fig:e}
%		\end{subfigure}%\hspace*{\fill}
%		\begin{subfigure}{0.43\textwidth}
%			\includegraphics[width=\linewidth]{images/data_i_NOISY3_0001_log.png}
%			%\caption{Fourth subfigure} \label{fig:f}
%		\end{subfigure}
%		\smallskip
%		\begin{subfigure}{0.95\textwidth}
%			\includegraphics[width=\linewidth]{images/legende_nomad.png}
%		\end{subfigure}
%		\caption{Comparaison sur problèmes bruités de Moré-Wild avec \CS} \label{fig:3}
%	\end{figure}
%	\clearpage
	Commentaires Commentaires Commentaires Commentaires Commentaires Commentaires Commentaires Commentaires Commentaires Commentaires Commentaires Commentaires Commentaires Commentaires Commentaires Commentaires Commentaires Commentaires Commentaires Commentaires Commentaires Commentaires Commentaires Commentaires Commentaires Commentaires Commentaires Commentaires Commentaires Commentaires Commentaires Commentaires Commentaires Commentaires Commentaires Commentaires Commentaires Commentaires Commentaires Commentaires Commentaires Commentaires              
\Chapter{CONCLUSION}\label{sec:Conclusion}
%%
%%  SYNTHESE DES TRAVAUX
%%
\section{Synthèse des travaux}
Ce travail comporte une comparaison de stratégies algorithmiques implémentables dans certains algorithmes d'optimisation sans-dérivées. Des algorithmes d'optimisation sans-dérivées identifiés pour la comparaison de ces stratégies, quatre sont des algorithmes de recherche directe directionnels, soient la recherche par coordonnée, la recherche par motifs généralisée, la recherche par ensemble générateur et la recherche directe par maillage adaptatifs. Un algorithme de recherche directe basée sur les dérivé du stencil, soit l'algorithme de filtrage implicite. 

La principale stratégie algorithmique étudiée, la stratégie opportuniste, consiste en l'arrêt prématuré d'une étape de l'algorithme dans laquelle il y a une succession d'évaluations de la fonction objectif à effectuer. La stratégie est implémentée dans les étapes de sonde des algorithmes identifiés préalablement. L'arrêt de la sonde est effectuée après l'évaluation d'un point dont la valeur de la fonction objectif à ce point est inférieure à la valeur minimale obtenue précédemment dans la résolution. La rritère d'arrêt peut être complexifié en y incorporant un nombre minimal de succès ou un nombre minimal d'évaluation.

L'arrêt prématuré implique que l'ordre d'évaluations séquentielle des points importe sur la performance. Plusieurs stratégies d'ordonnancement sont identifiées afin d'être comparées. Les stratégies sont l'ordonnancement lexicographique, l'ordonnancement aléatoire, l'ordonnancement en fonction de la direction du dernier succès et l'ordonnancement en fonction d'un modèle. À ces stratégies s'ajoutes les stratégies de comparaison, soient les stratégies omniscientes et inverse-omnisciente.

L'ensemble de ces stratégies algorithmiques est testé sur un ensemble de problèmes analytiques, un ensemble de problèmes contraints et la boîte noire STYRENE. Les tests numériques révèlent que, des différentes stratégies opportunistes, l'opportunisme après un seul succès est celle qui améliore le plus les méthodes. On y observe que l'impact de la stratégie opportuniste décroit avec le raffinement de l'étape de sonde de l'algorithme utilisé pour les méthodes directes directionnelles. Son impact sur la méthode de filtrage implicite est en apparence négatif. On y observe aussi que l'ordonnancement par modèle est la stratégie d'ordonnancement qui offre le meilleure rendement, suivi de la stratégie aléatoire et la stratégie de la direction du dernier succès. En général, on observe que l'opportunisme avec ordonnancement lexicographique nuit à la résolution. Les stratégies de comparaison permettent d'affirmer que les stratégies utilisées dans cette étude ne sont pas optimales et que l'opportunisme jumelé à une mauvaise stratégie d'ordonnancement peut nuire à la résolution de problèmes. Les résolutions de problèmes contraints montrent que l'intéraction entre la stratégie opportuniste et l'algorithme de la barrière progressive rends le classement des performances moins évident. Ce travail mène à la conclusion que l'opportunisme, jumelé à une stratégie d'ordonnancement adéquate, peut être bénéfique aux méthodes de recherche directe directionnelles, mais qu'il s'agit d'un outil à utiliser avec soin.
\section{Limitations de la solution proposée}\label{sec:Limitations}
Une boîte noire est un problème souvent long, bruité, non-différentiable et instable. Les problèmes analytiques proposés par Moré et Wild~\cite{MoWi2009} ne sont pas représentatifs des problèmes pour lesquels les algorithmes de résolution de boîte noire existent. Leurs aspects non-différentiables est limité et les amplitudes des bruits introduits sont faibles.

L'ensemble de problèmes contraints avec points de départ non-réalisables n'est pas suffisamment volumineux pour obtenir une tendance nette pour la performance de chaque stratégie d'ordonnancement identifiée sur cette catégorie de problème. L'ensemble de boîtes noires est limité à une seule, et les tendances sur ces problèmes complexes ne peuvent être validés ou infirmés sur d'autres boîtes noires.
\section{Améliorations futures}
Afin d'élargir les connaissances sur l'impact de la stratégie opportuniste en soit, il est envisageable d'identifier d'avantage de méthodes d'optimisation dans lesquelles elle pourrait être implémentée. Dans~\cite{JoPeSt93a}, les auteurs introduisent l'algorithme \textsf{DiRect}. Cet algorithme semble à première vue être un candidat qui possède les caractéristiques nécessaires pour l'implémentation de la stratégie. D'autres méthodes basées sur la recherche par coordonnées, notamment \textsf{PSwarm}~\cite{VaVi07}, figurent dans les algorithmes susceptibles d'être compatible avec la stratégie opportuniste.

Dans l'esprit de \GSS, un succès est accepté seulement si il entraîne une diminution suffisante de la fonction objectif. Ce critère de diminution suffisante pourrait être appliqué seulement comme critère d'opportunisme dans des algorithmes dont l'analyse de convergence ne nécessite pas la diminution suffisante. De cette façon, les succès marginaux seraient ignorés.

Les stratégies de comparaison montrent que le gain de performance de l'opportunisme pourrait être supérieur avec une stratégie d'ordonnancement plus efficace. Une utilisation future des modèles plus sophistiqués pour guider l'ordonnancement, tels que ceux proposés dans~\cite{AuKoLedTa2016}, serait à envisager. D'avantage de stratégie d'ordonnancement pourraient être identifiées. Dans \cite{CoScVibook}, les auteurs mentionnent une stratégie d'ordonnancement déterministe comme la stratégie lexicographique, qui consisterait en l'ordonnancement des points sans remise à zéro de l'ordre des points à chaque nouvelle sonde. De cette façon, aucune direction ne serait priorisée et la probabilité de choisir deux nouveaux centre de sonde issus de la même direction seraient nulles, ce que la stratégie aléatoire ne garanti pas. Les stratégies présentes dans les travaux de Custodio et Al.~\cite{CuDeVi08} pourraient être testées sur une série d'algorithmes.

Dans ce travail, l'ordonnancement en situation de barrière progressive est fixé à une dominance élaborée à la section~\ref{sec:omq}. Différentes relation d'ordre entre les points pourraient être étudiées pour déterminer les impacts de l'ordre d'évaluation de ces points en concert avec l'utilisation de la stratégie opportuniste.         % Conclusion.
%\backmatter
\renewcommand\bibname{RÉFÉRENCES}
\bibliography{bibliography}
%\bibliographystyle{polymtl}  % Format de la bibliographie.
\bibliographystyle{IEEEtranSN-francais}  % Format de la bibliographie.
%
\ifthenelse{\equal{\AnnexesPresentes}{O}}{
\appendix%
\newcommand{\Annexe}[1]{\annexe{#1}\setcounter{figure}{0}\setcounter{table}{0}\setcounter{footnote}{0}}%
%%
%%  Annexes.
%%
%%  Note: Ne pas modifier la ligne ci-dessous.
\addcontentsline{toc}{compteur}{ANNEXES}
%%
%%
%%  Toutes les annexes doivent être inclues dans ce document
%%  les unes à la suite des autres.
\Annexe{DÉMO}
Texte de l'annexe A\@. Remarquez que la phrase précédente se termine
par une lettre majuscule suivie d'un point. On indique explicitement
cette situation à \LaTeX{} afin que ce dernier ajuste correctement
l'espacement entre le point final de la phrase et le début de la
phrase suivante.


\begin{landscape}
\Annexe{ENCORE UNE ANNEXE}
Texte de l'annexe B\@ en mode «landscape».
\end{landscape}

\Annexe{UNE DERNIÈRE ANNEXE}
Texte de l'annexe C\@.
}{}
\end{document}
