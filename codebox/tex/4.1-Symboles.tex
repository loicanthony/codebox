% Liste des sigles et abbréviations
\newcommand\symbname{LISTE DES SYMBOLES}
\chapter*{\symbname}
\addcontentsline{toc}{compteur}{\symbname}
\pagestyle{pagenumber}
%
\begin{acronym}
  \acro{IETF}{Internet Engineering Task Force}
  \acro{OSI}{Open Systems Interconnection}
\end{acronym}
%
\begin{longtable}{lp{5in}}
$B$ & Matrice du modèle quadratique\\
$C^k$ & Matrice génératrice.\\
$c$ & Contrainte quantifiable\\
$D$ & Un ensemble générateur positif.\\
$d$ & Direction de recherche\\
$d_s$ & Profil de donnée du solveur $s$.\\
$e_i$ & Direction coordonnée.\\
$F^k$ & Cadre à l'itération $k$.\\
$f$ & Fonction objectif.\\
$f_{\Omega}$ & Fonction objectif sous barrière extrême.\\
$G$ & Matrice de base.\\
$H^k$ & Ensemble des directions supplémentaires pour la \textsf{SEARCH}.\\
$h$ & Quantification de la violation de contraintes\\
$k$ & Compteur d'itérations.\\
$L^k$ & Matrice de directions complémentaires à l'itération $k$.\\
$l$ & Borne inférieur\\
$M^k$ & Maillage à l'itération $k$.\\
$n$ & Dimension du problème.\\
$P^k$ & Ensemble des points de sonde à l'itération $k$. \\
$P$ & Ensemble de problèmes.\\
$p$ & Problème\\
$S^k$ & Ensemble des points de sonde à l'itération $k$. \\
$S$ & Ensemble de solveurs\\
$s$ & Solveur\\
$u$ & Borne supérieure\\
$x^k$ & Centre de sonde à l'itération $k$.\\

$\beta$ & Borne sur la longueur d'une direction.\\
$\Gamma^k$ & Matrice génératrice de l'espace.\\
$\Delta^k$ & Longueur du cadre à l'itération $k$\\
$\delta^k$ & Longueur du pas à l'itération $k$\\
$\epsilon_{\text{stop}}$ & Critère d'arrêt de l'algorithme\\
$\kappa(D^k)$ & Mesure cosinus de l'ensemble $D^k$.\\
$\lambda^k$ & Paramètre de contraction.\\
$\xi$ & Élément de la base naturelle.\\
$\rho(\delta^k)$ & Fonction de force.\\
$\rho_s$ & Profil de performance du solveur $s$.\\ 
$\tau$ & Paramètre d'ajustement du maillage.\\
$\phi^k$ & Paramètre d'expansion \\
$\Omega$ & Ensemble réalisable de solution\\
\end{longtable}

